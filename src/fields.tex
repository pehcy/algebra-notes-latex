\chapter{Fields}

\section{Extenstion Fields}

\subsection{Simple Extension}

\begin{example}
    $\mathbb{Q}[\sqrt{3}]$ is isomorphic to $\mathbb{Q}[\sqrt{3}]/\langle x^2 - 3 \rangle$
\end{example}

Let $\sigma: F \to E$ be an isomorphism then we again define 
$\olsi{\sigma} : F[x] \to E[x]$ by for $a_0 + a_1x + \cdots + a_n x^n \in F[x]$. We can write 
\begin{equation}
    \sigma (a_0 + a_1x + \cdots + a_n x^n) = \sigma(a_0) + \sigma(a_1x) + \cdots + \sigma(a_n x^n)
\end{equation}
and $\olsi{\sigma}$ is also isomorphism.

\begin{corollary}
    Let $\sigma : F \to E$ be an isomorphism of fields. Let $u$ be an algebraic element in 
    "some" extension field of $F$ with minimal polynomial $p(x) \in F[x]$. Again we let 
    $v$ be an algebraic element in some extension field of $E$ with minimal polynomial 
    $\sigma p(x) \in E[x]$. Then $\sigma$ extends to an isomorphism of fields 
    $\olsi{\sigma} : F(u) \to E(v)$ such that 
    \[
        \olsi{\sigma}(u) = v \> \text{ and } \> \olsi{\sigma}(c) = \sigma(c) \quad \forall c \in F.
    \]
\end{corollary}
\begin{proof}
    By previous theorem, $\varphi:F[x]/\left( p(x) \right) \to F(u)$ and
    $\olsi{\varphi}:E[x]/\left( \sigma p(x) \right) \to E(v)$ are isomorphism where 
    $\varphi(|f(x)|) = f(u)$ and $\olsi{\varphi}(|g(x)|) = g(v)$.

    Furthermore, we let $\xi $ be the surjective isomorphism
    \[
        \olsi{\xi}: E[x] \to E[x]/(\sigma p(x))
    \]
    defined by $\olsi{\xi}(g(x)) = |g(x)|$.
    
    Note that
    \[
        \begin{tikzcd}[row sep=2.5em]
            F[x] \arrow[r, "\sigma"] \arrow[dash]{d} & E[x] \arrow[r, "\olsi{\xi}"] \arrow[dash]{d} & E[x] / \left( \sigma p(x)\right) \arrow[r, "\olsi{\varphi}"] \arrow[dash]{d}& E[v] \arrow[dash]{d}\\
            f(x) \arrow[r] & \sigma f(x) \arrow[r] & \left| \sigma f(x) \right| \arrow[r] & \sigma f(v)\\
        \end{tikzcd}
    \]
    Since $\sigma, \olsi{\varphi}$ and $\olsi{xi}$ are surjective, so is the composite function.
    \begin{align*}
        ker\, \olsi{\phi}(\olsi{\xi}(\sigma)) &= 
        \{ f(x) \in F[x] \> | \> \sigma f(v) = 0 \}\\
        &= \{ f(x) \in F[x] \> | \> \sigma f(x) \in \langle \sigma f(x) \rangle \}\\
        &= \langle p(x) \rangle
    \end{align*}
    By First isomorphism theorem,
    \[
        \begin{tikzcd}[row sep=2.5em]
            & F \arrow[r, "\sigma"] \arrow[dash]{d} & E \arrow[dash]{d} & \\
            & F[x] \arrow[r, "\sigma"] \arrow[d, "\xi"] & E[x] \arrow[d, "\olsi{\xi}"] & \\  
            F(u) \arrow[r, "\varphi"] & F[x] / \langle p(x) \rangle \arrow[r, "\sigma"] & E[x] / \langle \sigma p(x)\rangle \arrow[r, "\olsi{\varphi}"] & E(v)\\
        \end{tikzcd}
    \]
\end{proof}


\subsection{Algebraic Extension}

\begin{definition}[Algebraic extension]
    An extension field $K$ of a field $F$ is said to be an algebraic extension of $F$ if every element 
    of $K$ is algebraic over $F$. 
\end{definition}

\begin{example}
    $\mathbb{C}$ is an algebraic extension of $\mathbb{R}$. $\forall a+bi \in \mathbb{C}$, where 
    $a,b \in \mathbb{R}$ and $i = \sqrt{-1}$. We have 
    \[
        (x+a+bi)(x+a-bi) = x^2 +2ax + a^2 + b^2.
    \]
    Thus $a+bi$ is a root of $x^2 +2ax + a^2 + b^2 = 0$.
\end{example}

\begin{theorem}
    If $K$ is a finite-dimensional extension field of $F$, then $K$ is an algebraic extension of $F$.
\end{theorem}
\begin{proof}
    Let $\{ V_1, V_2, \ldots, V_n \}$ be the basis of $K$ over $F$. For all $u \in K$, 
    $\{ 1, u, u^2, \ldots, u^n \}$ is linearly dependent. That is, 
    \[
        \exists u^k \in K \> s.t. \> u^k = \text{Span} \{1, u, u^2, \ldots, u^n\} = c_0 + c_1 u + c_2 u^2 +
        \cdots + c_{k-1} u^{k-1} (k \geq 1).
    \]
    Thus $u$ is a root of $f(x) = x^k - c_{k-1} u^{k-1} - \cdots - c_0$, this implies $K$ is an algebraic extension.
\end{proof}

\hrule
\vspace{10pt}

In fact, a simple extension is an algebraic extension if $u$ is algebraic. If extension field $K$ contains a 
transcental element $u$, then $K$ must be infinite dimensional over $F$.
\[
    \textsf{Non algebraic} \Longrightarrow \textsf{Infinite dimension}
\]
Note that $F(u)$ denote the intersection of all subfields of $K$ that contains both $F$ and $u$. It said to be a simple 
extension of $F$. If $u_1, u_2, \ldots, u_n$ are elements of an extension field $K$ of $F$. Let 
$F(u_1, \ldots, u_n)$ denote the intersection of all the subfields of $K$ that contain $F$ and every $u$ (known as 
generalized simple extension); $F(u, u_1, \ldots, u_n)$ is said to be a finitely generated extension of $F$.

\begin{theorem}
    If $K = F(u_1, u_2, \ldots, u_n)$ is a finitely generated extension field of 
    $F$ and each $u_i$ is algebraic over $F$, then $K$ is a finite-dimensional algebraic extension of $F$.
\end{theorem}
\begin{proof}
    Note that if $u, v$ is algebraic over $F$, then $v$ is algebraic over $F(u)$. Thus
    \[
        |F(u,v) : F(u)|\cdot |F(u) : F| < \infty \Longrightarrow 
        |F(u,v) : F| = |F(u,v) : F(u)| \cdot |F(u) : F| < \infty.
    \]
    By mathematical induction, we have 
    \[
        |F(u_1, u_2, \ldots, u_n) : F(u_1, u_2, \ldots, u_{n-1})| \ldots 
        |F(u_1) : F| < \infty
    \]
    which is also finite.
\end{proof}

\begin{corollary}
    If $L$ is algebraic extension of $K$ and $K$ is an algebraic extension of $L$. Then 
    $L$ is an algebraic extension field of $F$.
\end{corollary}
\begin{proof}
    $\forall \omega \in L, \> \exists f(x) \in K[x] \quad s.t. \> f(\omega) = a_0 + a_1 \omega 
    + \cdots + a_n \omega^n$. 

    Note that $F(a_0, a_1, \ldots, a_n)$ is finitely generated extension of $F$ and all $a_i$'s
    \\are algebraic. Thus it is finite dimensional algebraic extension of $F$. Since 
    $\omega$ is algebraic over $F(a_0, a_1, \ldots, a_n)$. So $F(a_0, a_1, \ldots, a_n)$ is finite 
    dimensional extension of $F \Longrightarrow \omega$ is algebraic over $F$.
    Thus $L$ is an algebraic extension of $F$.
\end{proof}

\begin{remark}
    Algebraic subfield $E$ of $\mathbb{C}$ over $\mathbb{Q}$ is called the \textbf{field of algebraic numbers}. 
    Where $E$ is an finite-dimensional algebraic extension over $K$.
    \[
        \begin{tikzcd}[row sep=tiny]
            \mathbb{C} &\\
             & E  \arrow[dash]{ul}
            \\
            \mathbb{Q} \arrow[uu, "\pi"] \arrow[ur, "\mu"]&\\
        \end{tikzcd}
    \]

    \begin{itemize}
        \item $\mu$ denote algebraic extension over $\mathbb{Q}$, e.g.: $\sqrt{2}, \sqrt{3}, i, \ldots$.
        \item $\pi$ denote non-algebraic extension.
    \end{itemize}
\end{remark}

\begin{corollary}
    Let $K$ be an extension field of $F$ and let $E$ be the set of all elements of $K$ that are 
    algebraic over $F$. Then $E$ is a subfield of $K$ and an algebraic extension field of $F$.
\end{corollary}
\begin{proof}
    We only need to show that $E$ is a field. Let $u,v \in F$, note that $F(u,v)$ 
    is finitely generated extension of $F$, so $E$ is algebraic extension.
    $E$ is closed under subtraction and multiplication. Moreover $u^{-1}$ is 
    algebraic over $F$. Thus $E$ is a subfield of $K$.
\end{proof}


\begin{example}
    \[
        \mathbb{Q}(i, -i) = \mathbb{Q}(i)
    \]
\end{example}

\begin{example}
    $$\mathbb{Q}(\sqrt{3}, i) = \mathbb{Q}(\sqrt{3})(i)$$
\end{example}
\begin{solution}
    \begin{align*}
        |\mathbb{Q}(\sqrt{3}, i)| &= |\mathbb{Q}(\sqrt{3})(i): \mathbb{Q}|\\
        &= |\mathbb{Q}(\sqrt{3})(i): \mathbb{Q}(\sqrt{3})| \, \cdot |\mathbb{Q}(\sqrt{3}): \mathbb{Q}|\\
        &= 2 \cdot 2\\
        &= 4
    \end{align*}
\end{solution}

\begin{example}
    Every finite-dimensional extension is also finitely generated. If $\{ u_1, u_2, \ldots, u_n \}$ 
    is a basis of $K$ over $F$. This implies $F(u_1, u_2, \ldots, u_n) \subseteq K$ and 
    $K \subseteq F(u_1, u_2, \ldots, u_n)$. 
    Thus,
    \[
        K = F(u_1, u_2, \ldots, u_n) = \text{Span}\{ u_1, u_2, \ldots, u_n \}.
    \]
\end{example}

\begin{example}[Non-example]
     $$\mathbb{Q}(\sqrt{3}, \sqrt{5}) \neq \mathbb{Q}(\sqrt{3})$$
\end{example}
\begin{solution}
    For the sake of contradiction, consider $\mathbb{Q}(\sqrt{3}, \sqrt{5}) = \mathbb{Q}(\sqrt{3})$, then 
    \[
        \sqrt{5} = a+ b\sqrt{3}, \quad \forall a,b \in \mathbb{Q}
    \]
    Altering this equation by moving $a$ to left-hand side, then squaring both sides. We obtain
    \begin{align*}
        (\sqrt{5} - a)^2 = (b\sqrt{3})^2 \quad &\Rightarrow 5 - 2\sqrt{5}a + a^2 = 3b^2\\
        &\Rightarrow \frac{5 + a^2 - 3b^2}{2a} = \sqrt{5} \quad (a\neq 0)
    \end{align*}
    However, when $a=0$, we have $5 = 3b^2$. Which is a contradiction.
\end{solution}

\section{Splitting Field}

In last chapter we had discussed about the integral domain. Suppose 
polynomial $f(x)$ has degree $n$. Then $f(x)$ has at most $n$ roots in \textit{any} field. 
Suppose that $K$ contains fewer than $n$ roots of $f(x)$. It might be possible to find an extension field of 
$K$ that contains additional roots of $f(x)$. 

\begin{definition}[Splitting field]
    If $F$ is a field and $f(x) \in F[x]$, then an extension field $K$ of $F$ is said to 
    be a \bred{splitting field} (or \bred{root field}) of $f(x)$ over $F$ provided that 
    \begin{itemize}
        \item $f(x)$ splits over $K$, say 
        \begin{equation}
            f(x) = c(x-u_1)\ldots (x-u_n)
        \end{equation}
        \item and
        \begin{equation}
            K = \underbrace{F(u_1, u_2, \ldots, u_n).}_{\text{smallest field}}
        \end{equation}
    \end{itemize}
\end{definition}

\begin{example}
    If $f(x) = x^4 - x^2 - 2 = (x^2 - 2)(x^2 + 1)$ in $\mathbb{Q}[x]$. Then 
    \[
        \mathbb{Q}(\sqrt{2}, -\sqrt{2}, i, -i) = \mathbb{Q}(\sqrt{2}, i)
    \]
    is a splitting field of $f(x)$ over $\mathbb{Q}$.
\end{example}


\section{Finite Fields}

\begin{theorem}
    Let $R$ be a ring with identity. Then 
    \begin{enumerate}
        \item The set 
        \[
            \mathfrak{P} = \{ k \cdot 1_R \> | \> k \in \mathbb{Z} \}
        \]
         is a subring of $R$.
        \item If $R$ has characteristic $0$, then $\mathfrak{P}$ is isomorphic to $\mathbb{Z}$.
        \item If $R$ has characteristic $n > 0$, then $\mathfrak{P}$ is isomorphic to $\mathbb{Z}_n$.
    \end{enumerate}
\end{theorem}
\begin{proof}
    We prove each of the statements listed above.
    \begin{enumerate}
        \item First of all, we use subring test to check if $\mathfrak{P}$ is a subring of $R$.
        \[
            \begin{cases}
                a \cdot 1_R - b \cdot 1_R = (a-b) \cdot 1_R \in \mathfrak{P}\\
                a \cdot 1_R b \cdot 1_R = ab\, 1_R \in \mathfrak{P}
            \end{cases}
        \]
        so $\mathfrak{P}$ is a subring of $R$. 
    \end{enumerate}

    We now prove (2), (3) at once, We consider a map $f : \mathbb{Z} \to \mathbb{R}$ 
    defined by 
    \[
        f(n) = n \cdot 1_R \quad \forall n \in \mathbb{Z}.
    \] 
    Then $f$ is homomorphism because 
    \[
        f(n + m) = (n + m) \cdot 1_R = f(n) + f(m)
    \]
    and the kernel is 
    \[
        ker\, f = \{ n \in \mathbb{Z} \> | \> n \cdot 1_R = 0_R \}.
    \]
    By the first isomorphism theorem, $\mathbb{Z} / ker\, f$ is isomorphic to $\mathbb{R}$.
    \begin{itemize}
        \item If $R$ has a characteristic $0$, then $ker\, f = \langle 0 \rangle \Longrightarrow \mathbb{Z} \cong \mathbb{R}$.
        \item If $R$ has a characteristic $n$, then $ker\, f = \langle n \rangle \Longrightarrow \mathbb{Z}/\langle n \rangle \cong \mathbb{R}$.
    \end{itemize}
\end{proof}

\subsection{Order of finite field}

\begin{theorem}
    A finite field $K$ has order $p^n$, where $p$ is the characteristic of $K$ 
    and $n = |K:\mathbb{Z}_p|$.
\end{theorem}
\begin{proof}
    Let $K$ be a finite dimensional extension of $\mathbb{Z}_p$. Let $n = |K:\mathbb{Z}_p|$, then 
    $\{ u_1, u_2, \ldots, u_n \}$ is a basis of $K$.

    $\forall k \in K$, $k$ is represented uniquely be 
    \[
        k = c_1 u_1 + c_2 u_2 + \cdots + c_nu_n.
    \]
    There are precisely $p^n$ distinct linear combinations of the form. Thus 
    $|K| = p^n$.
\end{proof}

\begin{lemma}[The Freshman's dream]
    Let $R$ be a commutative ring with identity of characteristic $p$, where $p$ is a prime.
    Then for every $a, b \in R$ and for all positive integer $n$ we have 
    \begin{equation}
        (a + b)^{p^n} = a^{p^n} + b^{p^n}.
    \end{equation}
\end{lemma}
\begin{proof}
    We will use the induction on $n$. 
    
    Assume $n = 1$, we expand $(a+b)^p$ with binomial theorem.
    \[
        (a+b)^p = \sum^p_{k=0} {p \choose k} a^{n-k}b^k = a^p + 
        {p \choose 1} a^{p-1}b + \cdots + {p \choose p-1} ab^{p-1} + b^p.
    \]
    Note that 
    \[
        {p \choose k} = \frac{p!}{(p-k)!\, k!}, \quad k, p-k < p \> \text{ for } 1 \leq k < p.
    \]
    This implies that $p$ divide ${p \choose k} \Longrightarrow 
    {p \choose k}a^{p-k}b^k = 0 \> (\text{mod } p)$. Thus $(a+b)^p = a^p + b^p$. 
    We are done for base case.

    Assume that it holds for all less than $n$.
    \begin{align*}
        (a+b)^{p^n} &= \left( (a+b)^p \right)^{p^{n-1}} \\
        &= (a^p + b^p)^{p^{n-1}}\\
        &= (a^p)^{p^{n-1}} + (b^p)^{p^{n-1}}\\
        &= a^{p^n} + b^{p^n}.
    \end{align*}

    Therefore the theorem is true for every positive integer $n$. Now we are done.
\end{proof}

\begin{theorem}[Existence of finite field]
    Let $K$ be an extension field $\mathbb{Z}_p$. For all positive integer $n$, 
    $K$ has order $p^n$ if and only if $K$ is a splitting field of $x^{p^n} - x$ over $\mathbb{Z}_p$.
\end{theorem}


\section{Fundamental Theorem of Galois theory}

\begin{definition}[Galois correspondence]
    Let $K$ be a finite-dimensionalextension field of $F$, and let 
    $S$ be the set of all intermediate fields. Again we let 
    $T$ be the set of all subgroups of the Galois group $\gal_F K$.

    Define a map $\phi : T \to S$ by this rule. For each intermediate field $E$,
    \begin{equation}
        \phi(E) = \gal_E K.
    \end{equation} 
    This function $\phi$ is called the Galois correspondence.

    \begin{center}
        $\begin{tikzcd}
            \gal_K K \arrow[r] 
            & K \arrow[d, "S"] \\
            \gal_F K \arrow[u,"T"] \arrow[r]
            & F
            \end{tikzcd}
        $
    \end{center}
\end{definition}

\begin{example}
    \[
        \mathbb{Q} \rightarrow \gal_\mathbb{Q} \mathbb{Q}(\sqrt{3}, \sqrt{5}) = \{i, \tau, \alpha, \beta \}
    \]
\end{example}

\begin{lemma}
    Let $K$ be a finite-dimensional extension field of $F$. If $H$ is a subgroup of the Galois group 
    $\gal_F K$ and $E$ is the \bgold{fixed field} of $H$, then $K$ is 
    \textit{simple, normal, separable extension} of $E$.

    \begin{center}
        $\begin{tikzcd}
            \gal_F K  
            & K \arrow[dash]{d} \\
            H \arrow[dash]{u} \arrow[r]
            & E_H
            \end{tikzcd}
        $
    \end{center}
\end{lemma}
\begin{proof}
    Since $K$ is finite-dimensional extension field, so $K$ is algebraic over $F$. Let $\mathfrak{U} \in K$ and 
    $p(x) \in E[x]$ be minimal polynomial of $\mathfrak{U}$, and $\forall \sigma \in H$, $\sigma(\mathfrak{U})$ is 
    some root of $p(x)$.

    Therefore, $\mathfrak{U}$ has a finite number of distinct images under automorphisms in $H$, said 
    \[
        \mathfrak{U} = u_1, u_2, \ldots, u_t \in K, \quad \text{where } t \leq \deg\, p(x)
    \]
    If $\sigma \in H$ and $u_i = \tau(\mathfrak{U})$ with $\tau \in H$, then $\sigma(u_i) = \sigma \circ \tau (\mathfrak{U})$.

    Since $\sigma$ is injective, so 
    \[
        \mathfrak{U} \xhookrightarrow{H} \{ u_1, u_2, \ldots, u_t \}
    \]
    which $\{ u_1, u_2, \ldots u_t\}$ is image of $\mathfrak{U}$. And $u_i = \tau(\mathfrak{U})$ for some $\tau \in H$.
    is injective 
    \[
        \{ u_1, u_2, \ldots, u_t \} \xhookrightarrow[\text{permutation } \sigma]{} \{ u_1, u_2, \ldots, u_t \}
    \]
    Every automorphism in $H$ permutes $u_1, u_2, \ldots, u_t$. Let 
    \[
        f(x) = (x-u_1)(x-u_2)\ldots(x-u_t)
    \]
    Since all $u_i$'s are distinct, $f(x)$ is separable.

    Now we claim that $f(x) \in E[x]$. Note that $\sigma f(x) = f(x)$ for all $\sigma \in H$. All coefficients of 
    $f(x)$ is fixed by $\sigma \in H$. Thus $f(x) \in E[x]$. Since $u = u_1$ is a root of $f(x) \in E[x]$, $u$ is separable over 
    $E. \Longrightarrow \> K$ is separable extension of $E$.

    We state that $K = E(V)$ for some $V \in K$. 
\end{proof}
