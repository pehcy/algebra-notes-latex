\chapter{Fields}

\section{Extenstion Fields}

\subsection{Simple Extension}

\begin{definition}
    Let $E$ be an extension field of a field $F$ and let $\mathfrak{a} \in E$. We call an element 
    $\mathfrak{a}$ \bred{algebraic} over $F$ if $\mathfrak{a}$ is the zero of some nonzero 
    polynomial in $F[x]$. If $\mathfrak{a}$ is not algebraic over $F$, it is called 
    \bred{transcendental} over $F$.
\end{definition}

\begin{example}
    $\mathbb{C}$ is an extension field of $\mathbb{R}$.
    \[
        \begin{tikzcd}[row sep=1.0em]
            \mathbb{C} \arrow[dash]{d}\\
            \mathbb{R}
        \end{tikzcd}
    \]
    The imaginary number $i = \sqrt{-1}$ is said to be algebraic since $x^2 + 1 = 0 \in \mathbb{R}[x]$. While 
    $\pi$ is transcendental since it is not a zero in $\mathbb{R}[x]$.
\end{example}

\begin{theorem}
    Let $K$ be an extension field of $F$, and $u \in K$ is an algebraic element over $F$. Then there exists 
    a unique monic irreducible polynomial $p(x)$ in $F[x]$ that has $u$ as a root. 
    Furthermore if $u$ is a root of $g(x) \in F[x]$, then $p(x)$ divide $g(x)$.
\end{theorem}
\begin{proof}
    Notice that $u \in K$ is algebraic over $F$ \textit{if and only if} there is a nonzero polynomial 
    $f(x) \in F[x]$ such that $f(u) = 0_K$.

    Let $S$ be the set of all nonzero polynomials in $F[x]$ that have $u$ as a root, then $S$ is nonempty set.
    By well-ordering principle, $\exists p(x) \in S$ such that $p(x)$ has the smallest degree in $S$.

    Suppose that $f(x) \in F[x]$ with $f(u) = 0_K$. By division algorithm, 
    \[
        f(x) = p(x) q(x) + r(x)
    \]
    with $\deg p(x) > \deg r(x)$ or $r(x) = 0$.

    If $r(x) \neq 0$, 
    \[
        f(x) - p(x) q(x) = r(x) \> \Longrightarrow f(u) - p(u) q(u) = r(u) = 0.
    \]
    This contradicting the fact that $p(x)$ is the smallest polynomial. Thus $r(x) = 0$ and 
    $p(x)$ divide $f(x)$. And now if we let $p(x)$ and $q(x)$ be the smallest polynomial. Then, 
    $p(x) | q(x)$ and $q(x) | p(x)$ implies that $p(x) = q(x)$.
\end{proof}

\begin{remark}
    The $p(x)$ is called the "minimal polynomial of $u$ over $F$".
\end{remark}

\begin{example}
    $\mathbb{R}$ is an extension field of $\mathbb{Q}$ and $\sqrt{3} \in \mathbb{R}$ is algebraic,
    then
    \[
        p(x) = x^2 - 3 \in \mathbb{Q}[x]
    \]
\end{example}

\begin{example}
    $\mathbb{R}$ is an extension field of $\mathbb{Q}$ and $\sqrt{3} + \sqrt{5} \in \mathbb{R}$ is algebraic,
    Then
    \[
        p(x) = x^4 - 16x^2 + 4 \in \mathbb{Q}[x].
    \]
\end{example}
\begin{solution}
    Let $x = \sqrt{3} + \sqrt{5} \in \mathbb{R}$, then 
    \begin{align*}
        x^2 = 3 + 2\sqrt{15} + 5 &\Rightarrow x^2 - 8 = 2\sqrt{15}\\
        &\Rightarrow (x^2 - 8)^2 = 4 \cdot 15\\
        &\Rightarrow x^4 - 16x^2 + 4 = 0
    \end{align*}
    Thus $p(x) = x^4 - 16x^2 + 4 \in \mathbb{Q}[x]$.
\end{solution}

\begin{theorem}
    Let $K$ be an extension field of $F$ and $u \in K$ is an algebraic element 
    over $F$ with minimal polynomial $p(x)$ of degree $n$, then 
    \begin{enumerate}
        \item $F[u]$ is a field isomorphism of $F[x]/p(x)$.
        \item $\{ 1, u, u^2, \ldots, u^{n-1}\}$ is a basis of the 
            vector space $F(u)$ over $F$.
        \item $|F(u):F|=n$.
    \end{enumerate}
\end{theorem}
\begin{proof}
    \begin{enumerate}
        \item Since $F(u)$ contains $F$ and $u$, so $F(u)$ contains every element of the form 
        \[
            b_0 + b_1u + b_2u^2 + \cdots + b_tu^t \quad \forall b_i \in F.
        \]
        We again define a function $\varphi : F[x] \to F(u)$ by 
        \[
            \varphi(f(x)) = f(u).
        \]
        Then $\varphi$ is ring homomorphism.

        Note that $ker\, \varphi = \langle p(x) \rangle$ where $p(x)$ is the minimal polynomial 
        of $u$ over $F$. By the first isomorphism theorem, $F[x] / p(x) \cong \Ima \varphi$.
        Since $p(x)$ is irreducible, the quotient ring $F[x] / \langle p(x) \rangle$ is a field and 
        $\Ima \varphi$ is also a field.

        Note that $\varphi(c) = c \> \forall c \in F$ and $\varphi (x) = u$. Thus $F \subset \Ima \varphi$ and 
        $u \in \Ima \varphi$. By definition of simple extension, $F(u) = \Ima \varphi$.

        \item Since $F(u) = \Ima \varphi, \> \forall w \in F(u), \exists f(x) \in F \> s.t. \>
        f(u) = w$. If $\deg f(x) > n$. By division algorithm, we have 
        \[
            f(x) = p(x) q(x) + r(x)
        \]
        If $r(x) = 0$, $f(u) = p(u)q(u) = 0 = w$.

        Otherwise, if $r(x) \neq 0$, then 
        \[
            r(x) = f(x) - p(x)q(x) \Longrightarrow r(u) = f(u) - p(u)q(u) = w
        \]
        with $\deg r(x) < n$. Thus 
        \[
            F(u) = \text{Span}\{1 , u, u^2, \ldots, u^{n-1} \}.
        \]
        
        \item Let $c_0 +c_1u + c_2u^2 + \cdots + c_{n-1}u^{n-1} = 0$. If 
        $\exists c_k \in F \> s.t. \> c_k \neq 0$. Then $p(x)$ is not the minimal polynomial of 
        $u$ over $F$. Thus $c_i = 0 \forall c_i \in F$. Hence, we can say that 
        $\{1, u, u^2, \ldots, u^{n-1} \}$ is linearly independent and it is also 
        a basis of $F(u)$ over $F \Longrightarrow |F(u):F| = n$.
    \end{enumerate}
\end{proof}

\begin{example}
    $\mathbb{Q}[\sqrt{3}]$ is isomorphic to $\mathbb{Q}[\sqrt{3}]/\langle x^2 - 3 \rangle$
\end{example}

\begin{example}
    If $u$ and $v$ have the same minimal polynomial $p(x)$ in $F[x]$, then $F(u)$ is 
    isomorphic to $F(v)$. For instance,
    \[
        \mathbb{Q}[\sqrt{3}] \cong \mathbb{Q}[-\sqrt{3}].
    \]
\end{example}

Let $\sigma: F \to E$ be an isomorphism then we again define 
$\olsi{\sigma} : F[x] \to E[x]$ by for $a_0 + a_1x + \cdots + a_n x^n \in F[x]$. We can write 
\begin{equation}
    \sigma (a_0 + a_1x + \cdots + a_n x^n) = \sigma(a_0) + \sigma(a_1x) + \cdots + \sigma(a_n x^n)
\end{equation}
and $\olsi{\sigma}$ is also isomorphism.

\begin{corollary}
    Let $\sigma : F \to E$ be an isomorphism of fields. Let $u$ be an algebraic element in 
    "some" extension field of $F$ with minimal polynomial $p(x) \in F[x]$. Again we let 
    $v$ be an algebraic element in some extension field of $E$ with minimal polynomial 
    $\sigma p(x) \in E[x]$. Then $\sigma$ extends to an isomorphism of fields 
    $\olsi{\sigma} : F(u) \to E(v)$ such that 
    \[
        \olsi{\sigma}(u) = v \> \text{ and } \> \olsi{\sigma}(c) = \sigma(c) \quad \forall c \in F.
    \]
\end{corollary}
\begin{proof}
    By previous theorem, $\varphi:F[x]/\left( p(x) \right) \to F(u)$ and
    $\olsi{\varphi}:E[x]/\left( \sigma p(x) \right) \to E(v)$ are isomorphism where 
    $\varphi( \left[ f(x) \right]) = f(u)$ and $\olsi{\varphi}( \left[ g(x) \right]) = g(v)$.

    Furthermore, we let $\xi $ be the surjective isomorphism
    \[
        \olsi{\xi}: E[x] \to E[x]/(\sigma p(x))
    \]
    defined by $\olsi{\xi}(g(x)) = |g(x)|$.
    
    Note that
    \[
        \begin{tikzcd}[row sep=2.5em]
            F[x] \arrow[r, "\sigma"] \arrow[dash]{d} & E[x] \arrow[r, "\olsi{\xi}"] \arrow[dash]{d} & E[x] / \left( \sigma p(x)\right) \arrow[r, "\olsi{\varphi}"] \arrow[dash]{d}& E[v] \arrow[dash]{d}\\
            f(x) \arrow[r] & \sigma f(x) \arrow[r] &  \left[ \sigma f(x) \right] \arrow[r] & \sigma f(v)\\
        \end{tikzcd}
    \]
    Since $\sigma, \olsi{\varphi}$ and $\olsi{xi}$ are surjective, so is the composite function.
    \begin{align*}
        ker\, \olsi{\phi}(\olsi{\xi}(\sigma)) &= 
        \{ f(x) \in F[x] \> | \> \sigma f(v) = 0 \}\\
        &= \{ f(x) \in F[x] \> | \> \sigma f(x) \in \langle \sigma f(x) \rangle \}\\
        &= \langle p(x) \rangle
    \end{align*}
    By First isomorphism theorem,
    \[
        F(u) \cong F[x] / \langle p(x) \rangle \cong^{\theta} E(v)
    \]
    Since $\theta([f(x)]) = \sigma f(v)$. Note that 
    \[
        \theta([x]) = \sigma \cdot 1_F \cdot v = 1_V \cdot v = v
    \]

    so we have the following situation
    \[
        \begin{tikzcd}[row sep=1.0em]
            F(u) & F[x] / \langle p(x) \rangle \arrow[l, "\varphi"] \arrow[r, "\theta"] & E(v)\\
            f(u) & \left[ f(x) \right] \arrow[l, "\varphi"] \arrow[r, "\theta"] & \sigma f(v)\\
            c & \left[ c \right] \arrow[l, "\varphi"] \arrow[r, "\theta"] & \sigma(c)\\
        \end{tikzcd}
    \]

    The composite function $\theta \circ \varphi^{-1} : F(u) \to E(v)$ is an isomorphism 
    that extends $\sigma$ and maps $u$ to $v$. 
    \[
        \tikz[overlay]{
      \draw (8.5,-2.25) node[anchor=south, color=cyan!70!black] {\textsf{By First isomorphism thm.}};
      \draw[ultra thick, cyan!70!black, fill=cyan!90!black, opacity=0.25, rounded corners] (5.35,-1.65) --
      (10,-1.65) --  (10,0.01) --  (7.35,0.01) --  (7.35,1.225) -- (2.82,1.225) -- (2.82,0.02) -- (5.35,0.02) -- cycle;
        }
        \begin{tikzcd}[row sep=2.5em]
            & F \arrow[r, "\sigma"] \arrow[dash, "\subseteq"]{d} & E \arrow[dash, "\subseteq"]{d} & \\
            & F[x] \arrow[r, "\sigma"] \arrow[d, "\xi"] & E[x] \arrow[d, "\olsi{\xi}"] & \\  
            F(u) \arrow[r, "\varphi"] & F[x] / \langle p(x) \rangle \arrow[r, "\sigma"] & E[x] / \langle \sigma p(x)\rangle \arrow[r, "\olsi{\varphi}"] & E(v)\\
        \end{tikzcd}
    \]
\end{proof}

\begin{example}
    $x^3 - 2$ is irreducible in $\mathbb{Q}[x]$ by Eisenstein's criterion. $\sqrt[3]{2} \in \mathbb{R}$ is a root of it.
    Verify that $\sqrt[3]{2} \omega$ is also a root of $x^3 - 2$ in $\mathbb{C}$ where 
    \[
        \omega = \frac{-1 + \sqrt{3}i}{2}
    \]
    is a complex cube root of $1$.
\end{example}
\begin{solution}
    Let $\sigma$ be the identity function from $\mathbb{Q}$ to $\mathbb{Q}$. By applying the previous corollary, we have
    \[
        \mathbb{Q}(\sqrt[3]{2}) \cong^\theta \mathbb{Q}(\sqrt[3]{2} \omega) 
    \]
    such that $\olsi{\sigma}(\sqrt[3]{2}) = \sqrt[3]{2} \omega$. And now $(\sqrt[3]{2} \omega)^3 = 2 \omega^3 = 2$.
\end{solution}


\subsection{Algebraic Extension}

\begin{definition}[Algebraic extension]
    An extension field $K$ of a field $F$ is said to be an algebraic extension of $F$ if every element 
    of $K$ is algebraic over $F$. 
\end{definition}

\begin{example}
    $\mathbb{C}$ is an algebraic extension of $\mathbb{R}$. $\forall a+bi \in \mathbb{C}$, where 
    $a,b \in \mathbb{R}$ and $i = \sqrt{-1}$. We have 
    \[
        (x+a+bi)(x+a-bi) = x^2 +2ax + a^2 + b^2.
    \]
    Thus $a+bi$ is a root of $x^2 +2ax + a^2 + b^2 = 0$.
\end{example}

\begin{theorem}
    If $K$ is a finite-dimensional extension field of $F$, then $K$ is an algebraic extension of $F$.
\end{theorem}
\begin{proof}
    Let $\{ V_1, V_2, \ldots, V_n \}$ be the basis of $K$ over $F$. For all $u \in K$, 
    $\{ 1, u, u^2, \ldots, u^n \}$ is linearly dependent. That is, 
    \[
        \exists u^k \in K \> s.t. \> u^k = \text{Span} \{1, u, u^2, \ldots, u^n\} = c_0 + c_1 u + c_2 u^2 +
        \cdots + c_{k-1} u^{k-1} (k \geq 1).
    \]
    Thus $u$ is a root of $f(x) = x^k - c_{k-1} u^{k-1} - \cdots - c_0$, this implies $K$ is an algebraic extension.
\end{proof}

\hrule
\vspace{10pt}

In fact, a simple extension is an algebraic extension if $u$ is algebraic. If extension field $K$ contains a 
transcental element $u$, then $K$ must be infinite dimensional over $F$.
\[
    \textsf{Non algebraic} \Longrightarrow \textsf{Infinite dimension}
\]
Note that $F(u)$ denote the intersection of all subfields of $K$ that contains both $F$ and $u$. It said to be a simple 
extension of $F$. If $u_1, u_2, \ldots, u_n$ are elements of an extension field $K$ of $F$. Let 
$F(u_1, \ldots, u_n)$ denote the intersection of all the subfields of $K$ that contain $F$ and every $u$ (known as 
generalized simple extension); $F(u, u_1, \ldots, u_n)$ is said to be a finitely generated extension of $F$.

\begin{theorem}
    If $K = F(u_1, u_2, \ldots, u_n)$ is a finitely generated extension field of 
    $F$ and each $u_i$ is algebraic over $F$, then $K$ is a finite-dimensional algebraic extension of $F$.
\end{theorem}
\begin{proof}
    Note that if $u, v$ is algebraic over $F$, then $v$ is algebraic over $F(u)$. Thus
    \[
        |F(u,v) : F(u)|\cdot |F(u) : F| < \infty \Longrightarrow 
        |F(u,v) : F| = |F(u,v) : F(u)| \cdot |F(u) : F| < \infty.
    \]
    By mathematical induction, we have 
    \[
        |F(u_1, u_2, \ldots, u_n) : F(u_1, u_2, \ldots, u_{n-1})| \ldots 
        |F(u_1) : F| < \infty
    \]
    which is also finite.
\end{proof}

\begin{corollary}
    If $L$ is algebraic extension of $K$ and $K$ is an algebraic extension of $L$. Then 
    $L$ is an algebraic extension field of $F$.
\end{corollary}
\begin{proof}
    $\forall \omega \in L, \> \exists f(x) \in K[x] \quad s.t. \> f(\omega) = a_0 + a_1 \omega 
    + \cdots + a_n \omega^n$. 

    Note that $F(a_0, a_1, \ldots, a_n)$ is finitely generated extension of $F$ and all $a_i$'s
    \\are algebraic. Thus it is finite dimensional algebraic extension of $F$. Since 
    $\omega$ is algebraic over $F(a_0, a_1, \ldots, a_n)$. So $F(a_0, a_1, \ldots, a_n)$ is finite 
    dimensional extension of $F \Longrightarrow \omega$ is algebraic over $F$.
    Thus $L$ is an algebraic extension of $F$.
\end{proof}

\begin{remark}
    Algebraic subfield $E$ of $\mathbb{C}$ over $\mathbb{Q}$ is called the \textbf{field of algebraic numbers}. 
    Where $E$ is an finite-dimensional algebraic extension over $K$.
    \[
        \begin{tikzcd}[row sep=tiny]
            \mathbb{C} &\\
             & E  \arrow[dash]{ul}
            \\
            \mathbb{Q} \arrow[uu, "\pi"] \arrow[ur, "\mu"]&\\
        \end{tikzcd}
    \]

    \begin{itemize}
        \item $\mu$ denote algebraic extension over $\mathbb{Q}$, e.g.: $\sqrt{2}, \sqrt{3}, i, \ldots$.
        \item $\pi$ denote non-algebraic extension.
    \end{itemize}
\end{remark}

\begin{corollary}
    Let $K$ be an extension field of $F$ and let $E$ be the set of all elements of $K$ that are 
    algebraic over $F$. Then $E$ is a subfield of $K$ and an algebraic extension field of $F$.
\end{corollary}
\begin{proof}
    We only need to show that $E$ is a field. Let $u,v \in F$, note that $F(u,v)$ 
    is finitely generated extension of $F$, so $E$ is algebraic extension.
    $E$ is closed under subtraction and multiplication. Moreover $u^{-1}$ is 
    algebraic over $F$. Thus $E$ is a subfield of $K$.
\end{proof}


\begin{example}
    \[
        \mathbb{Q}(i, -i) = \mathbb{Q}(i)
    \]
\end{example}

\begin{example}
    $$\mathbb{Q}(\sqrt{3}, i) = \mathbb{Q}(\sqrt{3})(i)$$
\end{example}
\begin{solution}
    \begin{align*}
        |\mathbb{Q}(\sqrt{3}, i)| &= |\mathbb{Q}(\sqrt{3})(i): \mathbb{Q}|\\
        &= |\mathbb{Q}(\sqrt{3})(i): \mathbb{Q}(\sqrt{3})| \, \cdot |\mathbb{Q}(\sqrt{3}): \mathbb{Q}|\\
        &= 2 \cdot 2\\
        &= 4
    \end{align*}
\end{solution}

\begin{example}
    Every finite-dimensional extension is also finitely generated. If $\{ u_1, u_2, \ldots, u_n \}$ 
    is a basis of $K$ over $F$. This implies $F(u_1, u_2, \ldots, u_n) \subseteq K$ and 
    $K \subseteq F(u_1, u_2, \ldots, u_n)$. 
    Thus,
    \[
        K = F(u_1, u_2, \ldots, u_n) = \text{Span}\{ u_1, u_2, \ldots, u_n \}.
    \]
\end{example}

\begin{example}[Non-example]
     $$\mathbb{Q}(\sqrt{3}, \sqrt{5}) \neq \mathbb{Q}(\sqrt{3})$$
\end{example}
\begin{solution}
    For the sake of contradiction, consider $\mathbb{Q}(\sqrt{3}, \sqrt{5}) = \mathbb{Q}(\sqrt{3})$, then 
    \[
        \sqrt{5} = a+ b\sqrt{3}, \quad \forall a,b \in \mathbb{Q}
    \]
    Altering this equation by moving $a$ to left-hand side, then squaring both sides. We obtain
    \begin{align*}
        (\sqrt{5} - a)^2 = (b\sqrt{3})^2 \quad &\Rightarrow 5 - 2\sqrt{5}a + a^2 = 3b^2\\
        &\Rightarrow \frac{5 + a^2 - 3b^2}{2a} = \sqrt{5} \quad (a\neq 0)
    \end{align*}
    However, when $a=0$, we have $5 = 3b^2$. Which is a contradiction.
\end{solution}

\section{Splitting Field}

In last chapter we had discussed about the integral domain. Suppose 
polynomial $f(x)$ has degree $n$. Then $f(x)$ has at most $n$ roots in \textit{any} field. 
Suppose that $K$ contains fewer than $n$ roots of $f(x)$. It might be possible to find an extension field of 
$K$ that contains additional roots of $f(x)$. 

\begin{definition}[Splitting field]
    If $F$ is a field and $f(x) \in F[x]$, then an extension field $K$ of $F$ is said to 
    be a \bred{splitting field} (or \bred{root field}) of $f(x)$ over $F$ provided that 
    \begin{itemize}
        \item $f(x)$ splits over $K$, say 
        \begin{equation}
            f(x) = c(x-u_1)\ldots (x-u_n)
        \end{equation}
        \item and
        \begin{equation}
            K = \underbrace{F(u_1, u_2, \ldots, u_n).}_{\text{smallest field}}
        \end{equation}
    \end{itemize}
\end{definition}

\begin{example}
    If $f(x) = x^4 - x^2 - 2 = (x^2 - 2)(x^2 + 1)$ in $\mathbb{Q}[x]$. Then 
    \[
        \mathbb{Q}(\sqrt{2}, -\sqrt{2}, i, -i) = \mathbb{Q}(\sqrt{2}, i)
    \]
    is a splitting field of $f(x)$ over $\mathbb{Q}$.
\end{example}


\section{Finite Fields}

\begin{theorem}
    Let $R$ be a ring with identity. Then 
    \begin{enumerate}
        \item The set 
        \[
            \mathfrak{P} = \{ k \cdot 1_R \> | \> k \in \mathbb{Z} \}
        \]
         is a subring of $R$.
        \item If $R$ has characteristic $0$, then $\mathfrak{P}$ is isomorphic to $\mathbb{Z}$.
        \item If $R$ has characteristic $n > 0$, then $\mathfrak{P}$ is isomorphic to $\mathbb{Z}_n$.
    \end{enumerate}
\end{theorem}
\begin{proof}
    We prove each of the statements listed above.
    \begin{enumerate}
        \item First of all, we use subring test to check if $\mathfrak{P}$ is a subring of $R$.
        \[
            \begin{cases}
                a \cdot 1_R - b \cdot 1_R = (a-b) \cdot 1_R \in \mathfrak{P}\\
                a \cdot 1_R \cdot b \cdot 1_R = ab\, 1_R \in \mathfrak{P}
            \end{cases}
        \]
        so $\mathfrak{P}$ is a subring of $R$. 
    \end{enumerate}

    We now prove (2), (3) at once, We consider a map $f : \mathbb{Z} \to \mathbb{R}$ 
    defined by 
    \[
        f(n) = n \cdot 1_R \quad \forall n \in \mathbb{Z}.
    \] 
    Then $f$ is homomorphism because 
    \[
        f(n + m) = (n + m) \cdot 1_R = f(n) + f(m)
    \]
    and the kernel is 
    \[
        ker\, f = \{ n \in \mathbb{Z} \> | \> n \cdot 1_R = 0_R \}.
    \]
    By the first isomorphism theorem, $\mathbb{Z} / ker\, f$ is isomorphic to $\mathbb{R}$.
    \begin{itemize}
        \item If $R$ has a characteristic $0$, then $ker\, f = \langle 0 \rangle \Longrightarrow \mathbb{Z} \cong \mathbb{R}$.
        \item If $R$ has a characteristic $n$, then $ker\, f = \langle n \rangle \Longrightarrow \mathbb{Z}/\langle n \rangle \cong \mathbb{R}$.
    \end{itemize}
\end{proof}

\subsection{Order of finite field}

\begin{theorem}
    A finite field $K$ has order $p^n$, where $p$ is the characteristic of $K$ 
    and $n = |K:\mathbb{Z}_p|$.
\end{theorem}
\begin{proof}
    Let $K$ be a finite dimensional extension of $\mathbb{Z}_p$. Let $n = |K:\mathbb{Z}_p|$, then 
    $\{ u_1, u_2, \ldots, u_n \}$ is a basis of $K$.

    $\forall k \in K$, $k$ is represented uniquely be 
    \[
        k = c_1 u_1 + c_2 u_2 + \cdots + c_nu_n.
    \]
    There are precisely $p^n$ distinct linear combinations of the form. Thus 
    $|K| = p^n$.
\end{proof}

\begin{lemma}[The Freshman's dream]
    Let $R$ be a commutative ring with identity of characteristic $p$, where $p$ is a prime.
    Then for every $a, b \in R$ and for all positive integer $n$ we have 
    \begin{equation}
        (a + b)^{p^n} = a^{p^n} + b^{p^n}.
    \end{equation}
\end{lemma}
\begin{proof}
    We will use the induction on $n$. 
    
    Assume $n = 1$, we expand $(a+b)^p$ with binomial theorem.
    \[
        (a+b)^p = \sum^p_{k=0} {p \choose k} a^{n-k}b^k = a^p + 
        {p \choose 1} a^{p-1}b + \cdots + {p \choose p-1} ab^{p-1} + b^p.
    \]
    Note that 
    \[
        {p \choose k} = \frac{p!}{(p-k)!\, k!}, \quad k, p-k < p \> \text{ for } 1 \leq k < p.
    \]
    This implies that $p$ divide ${p \choose k} \Longrightarrow 
    {p \choose k}a^{p-k}b^k = 0 \> (\text{mod } p)$. Thus $(a+b)^p = a^p + b^p$. 
    We are done for base case.

    Assume that it holds for all less than $n$.
    \begin{align*}
        (a+b)^{p^n} &= \left( (a+b)^p \right)^{p^{n-1}} \\
        &= (a^p + b^p)^{p^{n-1}}\\
        &= (a^p)^{p^{n-1}} + (b^p)^{p^{n-1}}\\
        &= a^{p^n} + b^{p^n}.
    \end{align*}

    Therefore the theorem is true for every positive integer $n$. Now we are done.
\end{proof}

\begin{theorem}[Existence of finite field]
    Let $K$ be an extension field $\mathbb{Z}_p$. For all positive integer $n$, 
    $K$ has order $p^n$ if and only if $K$ is a splitting field of $x^{p^n} - x$ over $\mathbb{Z}_p$.
\end{theorem}

\section{Galois Theory}

\subsection{Fundamental Theorem of Galois theory}

\begin{definition}[Galois correspondence]
    Let $K$ be a finite-dimensionalextension field of $F$, and let 
    $S$ be the set of all intermediate fields. Again we let 
    $T$ be the set of all subgroups of the Galois group $\gal_F K$.

    Define a map $\phi : T \to S$ by this rule. For each intermediate field $E$,
    \begin{equation}
        \phi(E) = \gal_E K.
    \end{equation} 
    This function $\phi$ is called the Galois correspondence.

    \begin{center}
        $\begin{tikzcd}
            \gal_K K \arrow[r] 
            & K \arrow[d, "S"] \\
            \gal_F K \arrow[u,"T"] \arrow[r]
            & F
            \end{tikzcd}
        $
    \end{center}
\end{definition}

\begin{example}
    \[
        \mathbb{Q} \rightarrow \gal_\mathbb{Q} \mathbb{Q}(\sqrt{3}, \sqrt{5}) = \{i, \tau, \alpha, \beta \}
    \]
\end{example}

\begin{lemma}
    Let $K$ be a finite-dimensional extension field of $F$. If $H$ is a subgroup of the Galois group 
    $\gal_F K$ and $E$ is the \bgold{fixed field} of $H$, then $K$ is 
    \textit{simple, normal, separable extension} of $E$.

    \begin{center}
        $\begin{tikzcd}
            \gal_F K  
            & K \arrow[dash]{d} \\
            H \arrow[dash]{u} \arrow[r]
            & E_H
            \end{tikzcd}
        $
    \end{center}
\end{lemma}
\begin{proof}
    Since $K$ is finite-dimensional extension field, so $K$ is algebraic over $F$. Let $\mathfrak{U} \in K$ and 
    $p(x) \in E[x]$ be minimal polynomial of $\mathfrak{U}$, and $\forall \sigma \in H$, $\sigma(\mathfrak{U})$ is 
    some root of $p(x)$.

    Therefore, $\mathfrak{U}$ has a finite number of distinct images under automorphisms in $H$, said 
    \[
        \mathfrak{U} = u_1, u_2, \ldots, u_t \in K, \quad \text{where } t \leq \deg\, p(x)
    \]
    If $\sigma \in H$ and $u_i = \tau(\mathfrak{U})$ with $\tau \in H$, then $\sigma(u_i) = \sigma \circ \tau (\mathfrak{U})$.

    Since $\sigma$ is injective, so 
    \[
        \mathfrak{U} \xhookrightarrow{H} \{ u_1, u_2, \ldots, u_t \}
    \]
    which $\{ u_1, u_2, \ldots u_t\}$ is image of $\mathfrak{U}$. And $u_i = \tau(\mathfrak{U})$ for some $\tau \in H$.
    is injective 
    \[
        \{ u_1, u_2, \ldots, u_t \} \xhookrightarrow[\text{permutation } \sigma]{} \{ u_1, u_2, \ldots, u_t \}
    \]
    Every automorphism in $H$ permutes $u_1, u_2, \ldots, u_t$. Let 
    \[
        f(x) = (x-u_1)(x-u_2)\ldots(x-u_t)
    \]
    Since all $u_i$'s are distinct, $f(x)$ is separable.

    Now we claim that $f(x) \in E[x]$. Note that $\sigma f(x) = f(x)$ for all $\sigma \in H$. All coefficients of 
    $f(x)$ is fixed by $\sigma \in H$. Thus $f(x) \in E[x]$. Since $u = u_1$ is a root of $f(x) \in E[x]$, $u$ is separable over 
    $E. \Longrightarrow \> K$ is separable extension of $E$.

    We state that $K = E(V)$ for some $V \in K$. 
\end{proof}

\section{Solvability by Radicals}

We shall assume that all fields have characteristic $0$. A "formula" is a specific procedure 
that starts with coefficients of the polynomial $f(x) \in F[x]$ and arrives at the solutions of equation
$f(x) = 0_F$ by using only the field operations $(+_F, -_F, \times_F, \div_F)$ and the extraction 
of roots (such as $\sqrt[n]{\cdot}$).

In this context, an $n$-th root of an element $c$ in field $F$ is any root of the polynomial 
$x^n - c$ in some extension field of $F$. If that "formula" really exists, then there exists 
an extension field $K$ of $F$ such that 
\[
    \begin{tikzcd}
        F \arrow[r, blue, bend left=50, "\text{splitting field of } x^n - c"] & F' \arrow[r, red, bend left=50, "\text{splitting field of } x^{n'} - c'"]& \ldots \arrow[r, bend left=50]& K\\
    \end{tikzcd}
\]

\begin{theorem}
    For $n \geq 5$ the group $S_n$ is not solvable.
\end{theorem}
\begin{proof}
    For the sake of contradiction, suppose that 
    $S_n$ is solvable and that 
    \[
        S_n = G_0 \supseteq G_1 \supseteq \cdots \supseteq G_t = \langle 1 \rangle.
    \] 
    Let $(r\, s\, t)$ be any 3-cycle in $S_n$ and let $\alpha, \beta$ be any 
    element of $\{1, 2, \ldots, n\}$ other than $r, s, t$ (they always exist since $n \geq 5$).
    Since $S_n / G_1$ is abelian, by theorem of dihedral group,
    \begin{align*}
        (t\, \alpha\, s)(s\, r\, \beta)(t\, \alpha\, s)^{-1} (s\, r\, \beta)^{-1} &= 
        (t\, \alpha\, s)(s\, r\, \beta)(t\, s\, \alpha)(s\, \beta\, r)\\
        &= (r\, s\, t) \in G_1
    \end{align*}
    Note that the cycle $\langle (r\, s\, t) \rangle \subseteq G_1$, and $G_1$ definitely contains all the 3-cycles 
    since $G_1/G_2$ is abelian, repeating upper process, $G_2$ also contains $\langle (r\, s\, t) \rangle$.
    
    In conclusion, $\forall i \in \{1,2,\ldots, n\}$, $G_i$ contains all the 3-cycles. This contradicting the fact 
    that $S_n$ is solvable.
\end{proof}

\begin{definition}
    A generator of this cyclic group of $n$-th roots of unity in $K$ is called a primitive $n$-th root of unity.
\end{definition}

This definition states that $\zeta$ is a primitive $n$-th roots of unity iff $\zeta, \zeta^2, \ldots, \zeta^n$ 
are the $n$ distinct $n$-th roots of unity.

\begin{example}
    Consider $x^4 - 1 \in \mathbb{Q}[x]$. The 4-th roots of unity in $\mathbb{C} = \{1, -1, i, -1\} = \langle i \rangle$.
    $i$ and $-i$ are primitive 4-th root of unity in $\mathbb{C}$.
\end{example}

\begin{example}
    According to De Moivre's theorem,
    \[
        \cos \left(\frac{2 \pi}{n}\right) + i \sin \left(\frac{2 \pi}{n}\right)
    \]
    is a primitive $n$-th root of unity in $\mathbb{C}$.
\end{example}