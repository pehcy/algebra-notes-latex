\chapter{Modules and Vector Spaces}

Modules are essentially the equivalent of vector spaces when the scalars come from a ring rather than a field.
Throughout this notes we will consider only modules over Euclidean domains. 

\begin{axiom}[Modules]
    Let $R$ be a ring. A left $R$-module $M$ consists of an abelian group $(M, +)$, together 
    with scalar multiplication $\cdot : R \times M \mapsto M$ such that 
    \begin{itemize}
        \item $r(m_1 + m_2) = rm_1 + rm_2$,\quad $\forall r \in R$ and $m_1, m_2 \in M$.
        \item $(r_1 + r_2)m = r_1m + r_2m$, \quad $\forall r_1, r_2 \in R$ and $m \in M$.
        \item $(r_1 r_2)m = r_1 (r_2 m)$, \quad $\forall r_1, r_2 \in R$ and $m \in M$. 
        \item $1_R \cdot m = m$, \quad $\forall m \in M$.
    \end{itemize}
\end{axiom}

If $R$ is a field, then the $R$-module is called a \textbf{vector space} over $R$.

\begin{definition}[Vector Space (defn. from linear algebra)]
    A \bred{vector space} is a nonempty set $V$ of objects, known as \textit{vectors}, on which 
    are defined by two operations: \textit{addition} and \textit{scalar multiplication} in $\mathbb{R}^n$.
    The properties of modules must hold for all $\mathbf{u}, \mathbf{v}, \mathbf{w} \in V$; and for all 
    $c, d \in \mathbb{R}^n$.
    \begin{itemize}
        \item $\mathbf{u}, \mathbf{v} \in V$.
        \item $(\mathbf{u} + \mathbf{v}) + \mathbf{w} = \mathbf{u} + (\mathbf{v} + \mathbf{w})$.
        \item $c\mathbf{u} \in V$.
        \item $(cd)\mathbf{u} = c(d\mathbf{u})$.
        \item $1 \mathbf{u} = \mathbf{u}$.
    \end{itemize}
\end{definition}

\begin{definition}
    Let $R$ be a ring, and let $M$ be an $R$-module. 
    \begin{enumerate}
        \item We say that elements $x_1, \dots, x_n \in M$ are \bred{linearly independent} if the only way to express the zero element as a linear combination of these elements is with all coefficients equal to zero.
        \item We say that elements $x_1, \dots, x_n \in M$ are a \bred{basis} for $R$-module $M$ if they are linearly independent and generate $M$.
        \item We say that element $x_1, \dots, x_n$ of an $R$-module $M$ \bred{generate} or \bred{span} $M$ if every element of $M$ can be expressed as a linear combination of $x_1, \dots, x_n$. That is,
        \begin{equation}
            \overline{x} = \text{Span} \{x_1, \dots, x_n \} = r_1x_1 + r_2x_2 + \cdots + r_nx_n
        \end{equation}
        for some $r_1, r_2, \ldots, r_n \in R$.
    \end{enumerate}
\end{definition}

\begin{theorem}
    A submodule of a free module is free.
\end{theorem}
\begin{proof}
    Let $R$ be a PID. Suppose $P = \bigoplus_{j \in J} R_j$ be a free $R$-module with basis $J$, 
    and suppose $M \subset P$ is a submodule.

    Construct a well-ordering of the set $j$. $\forall j \in J$, we set $P_j := \bigoplus_{i \leq j} R_i$ 
    and $\overline{R}_i$, so that $P_j = \overline{P}_j \oplus R$.

    Let $\phi_i$ be the composite 
    \begin{equation}
        P_i \cap M\hookrightarrow P_j = \overline{P}_j \oplus R \leadsto R.
    \end{equation}
    The $Ker \phi_i = \overline{P}_j \cap M$, and $\Ima \phi_i \subset R$ is an ideal. Hence let 
    \[
        \Ima \phi_i = (\lambda_j) \quad \text{ for some } \lambda_j \in R.
    \]
    Pick some $c_j \in P_i \cap M$ such that $\phi(c_j) = \lambda_j$. Let
    \begin{equation}
        J' = \{ j \in J \> | \> \lambda_j \neq 0 \}.
    \end{equation}
    To complete the proof.

    \textbf{[Claim]}: $\{ c_j \}_{j \in J'}$ is a basis.

    Suppose that $\{ c_j \}_{j \in J'}$ is linearly independent.
    \begin{equation}
        \sum^n_{k=1} a_kc_{j_k} = 0
    \end{equation}
\end{proof}