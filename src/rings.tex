\chapter{Rings}

Ring is an algebraic structure which is a set of elements with two operations: addition and multiplication.

\begin{axiom}[Rings]
    A ring $R$ is a set with two binary operation, addition (usually denoted by $+$) and multiplication 
    (usually denoted by $ab$), such that for all $a,b,c \in R$.
    \begin{enumerate}
        \item Addition is commutative, $a + b = b + a$.
        \item Associativity holds in addition, $(a + b) + c = a + (b + c)$.
        \item There is an additive identity $0_R$. That is, there is an element $0_R \in R$ such that 
        \[
            a + 0_R = a = 0_R + a
        \]
        for all $a \in R$.
        \item There is an additive inverse $-a \in R$ such that
        \[
            a + (-a) = 0_R = -a + a.
        \]
        \item Associativity holds in multiplication, $a(bc)=(ab)c$.
        \item Distributive law holds in $R$, 
        \[
            a(b+c) = ab + ac 
        \]
        and 
        \[
            (b+c)a = ba + ca.
        \]
    \end{enumerate}
\end{axiom}

Here are a few things you should take notes:
\begin{enumerate}
    \item A ring is an Abelian group under addition, also having an associative 
    multiplication that is left and right distributive over addition. 
     
    \item Note that multiplication need not be commutative. When it is, we say that the 
    ring is commutative. 
     
    \item A ring need not have an identity under multiplication. A unity (or identity) in a 
    ring is a nonzero element that is an identity under multiplication. 
     
    \item A nonzero element of a commutative ring with unity need not have 
    multiplicative inverse. When it does, we say that it is a unit of the ring. Thus, 
    $x$ is a unit if $x^{-1}$ exists.
     
    \item We follow the following terminology and notation. If $x$ and $y$ belong to a 
    commutative ring $R$ and $x$ is nonzero, we say that $x$ divides $y$ and write $x | y$, 
    if there exists an element $c$ in $R$ such that $y = xc$. 
     
    \item If $x$ is an element from a group under the operation of addition and $n$ is a 
    positive integer, $nx$ means $\underbrace{x + x + \cdots + x}_{n \text{ times}}$, where there are $n$ summands. 
\end{enumerate}

\begin{example}
    The set $\mathbb{Z}$ of integers under ordinary addition and multiplication is a commutative ring with unity $1$. The units 
    of $\mathbb{Z}$ are $1$ and $-1$.
\end{example}

\begin{example}
    The set $M_2(\mathbb{Z})$ of $2 \times 2$ matrices with integer entries is a noncommutative ring with unity 
    $\begin{bmatrix}
        1 & 0\\ 0 & 1
    \end{bmatrix}$.
\end{example}

\begin{example}
    The set of all continuous real-valued functions of a real variable whose graphs pass through 
    the point $(1, 0)$ is a commutative ring without unity under the operations of pointwise addition and multiplication, that is, 
    \[
        (f + g)(x) = f(x) + g(x)
    \]
    and 
    \[
        (fg)(x) = f(x)g(x).
    \]
\end{example}

\begin{theorem}
    Let $a,b$ and $c$ belong to a ring $R$. Then 
    \begin{enumerate}
        \item $a0 = 0a = 0$.
        \item $a(-b) = (-a)b = -(ab)$.
        \item $(-a)(-b) = ab$.
        \item $a(b-c) = ab - ac$ and $(b-c)a = ba-ca$.
        \item Furthermore, if $R$ has a unity element 1, then $(-1)a = -a$.
        \item $(-1)(-1) = (-1)^2 = 1$.
    \end{enumerate}
\end{theorem}
\begin{proof}
    We are going to prove each of them:

    \begin{enumerate}
        \item \begin{align*}
            a(0 + 0) = a0 &\Rightarrow a0 + a0 = a0\\
            &\Rightarrow {\color{red}{} (-a0)} + a0 + a0 = {\color{red}{} (-a0)} + a0\\
            &\Rightarrow 0 + a0 = 0\\
            &\Rightarrow a0 = 0.
        \end{align*}

        \item \begin{align*}
            a0 = 0 &\Rightarrow a(b + (-b)) = 0\\
            &\Rightarrow ab + a(-b) = 0\\
            &\Rightarrow {\color{red}{} -(ab)} + ab + a(-b) = {\color{red}{} -(ab)} + 0\\
            &\Rightarrow a(-b) = -(ab).
        \end{align*}

        \item \begin{align*}
            (-a)(-b) &= -(a(-b))\\
            &= -(a(-b))\\
            &= -(-(ab))\\
            &= ab.
        \end{align*}

        \item \[ 
            a(b-c) = a(b+(-c)) = ab + a(-c) = ab - ac.
        \]

        \item \begin{align*}
            0a = 0 &\Rightarrow (1+(-1))a = 0\\
            &\Rightarrow 1a + (-1)a = 0\\
            &\Rightarrow a + (-1)a = 0\\
            &\Rightarrow {\color{red}{} (-a)} + a + (-1)a = {\color{red}{} (-a)} + 0\\
            &\Rightarrow 0 + (-1)a =  {\color{red}{} (-a)}\\
            &\Rightarrow (-1)a = -a.
        \end{align*}

        \item \[
            (-1)(-1) = -(1(-1)) = -(-(1 \cdot 1)) = 1.
        \]
    \end{enumerate}
\end{proof}

\begin{lemma}
    If a ring has a unity, it is unique.
\end{lemma}
\begin{proof}
    Suppose $1_R$ and $1'_R$ are both unity of ring $R$. Then 
    For all $a \in R$, we must have 
    \[
        a \cdot 1_R = a = 1_R \cdot a
    \]
    and 
    \[
        a \cdot 1'_R = a = 1'_R \cdot a.
    \]
    Thus, we say that $a \cdot 1_R = a = a \cdot 1'_R$. By cancellation law this yields
    \[
        1_R = 1'_R.
    \]
    So ring $R$ has unique unity.
\end{proof}

\section{subrings}

\begin{theorem}[Subring test]
    A nonempty subset $S$ of the ring $R$ is a subring if $S$ is closed under subtraction and multiplication, that is,
    \begin{enumerate}
        \item $S \neq \varnothing$.
        \item $a - b \in S \quad \forall a, b \in S$.
        \item $ab \in S \quad \forall a, b \in S$.
    \end{enumerate}
\end{theorem}
\begin{proof}
    Since addition in $R$ is commutative and $S$ closed under subtraction, we know that by 
    the \textit{one-step subgroup test} that $S$ is an Abelian group under addition. 
    
    Also, since multiplication in $R$ is asscoiative as well as distributive over addition, the 
    same is true for multiplication in $S$.

    Therefore, the only condition remaining to be checked is that multiplication is a binary 
    operation on $S$. But this is exactly what closure means.
\end{proof}

\begin{example}
    $S = \{ 0, 2, 4 \}$ is a subring of $\mathbb{Z}_6$.
\end{example}
\begin{solution}
    Using subring test, the subtraction and multiplication form a group in $S$.

    \begin{center}
        \begin{minipage}{.25\textwidth}
            {
            \arrayrulewidth=1pt
            \renewcommand{\arraystretch}{1.5}
            \begin{tabular}{c|*{3}{>{\columncolor{white}}c}}
              $-$ & \cellcolor{myred}0 & \cellcolor{mygreen}2 & \cellcolor{mypurple}4 \\
              \hline
              \cellcolor{myred}0 & \cellcolor{myred}0 & \cellcolor{mypurple}4 & \cellcolor{mygreen}2 \\
              \cellcolor{mygreen}2 & \cellcolor{mygreen}2 & \cellcolor{myred}0 & \cellcolor{mypurple}4\\
              \cellcolor{mypurple}4 & \cellcolor{mypurple}4 & \cellcolor{mygreen}2 & \cellcolor{myred}0\\
            \end{tabular}
            }
          \end{minipage}% This must go next to `\end{minipage}`
          \begin{minipage}{.25\textwidth}
            {
            \arrayrulewidth=1pt
            \renewcommand{\arraystretch}{1.5}
            \begin{tabular}{c|*{3}{>{\columncolor{white}}c}}
              $\cdot$ & \cellcolor{myred}0 & \cellcolor{mygreen}2 & \cellcolor{mypurple}4 \\
              \hline
              \cellcolor{myred}0 & \cellcolor{myred}0 & \cellcolor{myred}0 & \cellcolor{myred}0 \\
              \cellcolor{mygreen}2 & \cellcolor{myred}0 & \cellcolor{mypurple}4 & \cellcolor{mygreen}2\\
              \cellcolor{mypurple}4 & \cellcolor{myred}0 & \cellcolor{mygreen}2 & \cellcolor{mypurple}4\\
            \end{tabular}
            }
          \end{minipage}
    \end{center}

    Thus $S \leq \mathbb{Z}_6$.
\end{solution}

\section{Quotient rings, Ideals}

\subsection{Ideals}

\begin{definition}[Ideals]
    A subring $A$ of a ring $R$ is called an ideal if for every $r \in R$ and every $a \in A$ both $ra$ and 
    $ar$ are in $A$.
\end{definition}

So, a subring $A$ of a ring $R$ is an ideal of $R$ if 
\[
   rA = \{ ra \> | \> a \in A \} \subseteq A
\]
and 
\[
    Ar = \{ ar \> | \> a \in A \} \subseteq A
\]
for all $r \in R$. An ideal $A$ of $R$ is called a \bred{proper ideal} of $R$ if $A$ is a proper subset of $R$.

\begin{lemma}
    Let $I$ be a subring of a ring $R$. Then $I$ is an ideal in $R$ if and only if multiplication 
    \[
        (a + I)(b + I) = (ab + I)
    \]
    is a well-defined operation on the cosets of $I$ in $R$.
\end{lemma}
\begin{proof}
    $(\Rightarrow)$ Assume that $I$ is an ideal in $R$, and suppose 
    that $a_1 + I = a_2 + I$ and $b_1 + I = b_2 + I$. This implies that $a_1 = a_2 + k$ and 
    $b_1 = b_2 + j$ for some $i, j \in I$. Then we have 
    \[
        a_1 b_1 = a_2 b_2 + a_2 j + kb_2 + kj.
    \]
    Since $I$ is a subring of $R$, and therefore it closed under multiplication, as well as addition, and $kj \in I$. 
    Since $I$ is an ideal, $a_2 j \in I$ and $kb_2 \in I$, and so $a_2 j + kb_2 + jk \in I$. 

    Therefore, $a_1b_1 \in a_2b_2 + I$ and $a_1b_1 + I = a_2b_2 + I$. Thus, the multiplication on the set 
    of cosets of $I$ is well-defined.

    $(\Leftarrow)$ Assume that the indicated operation is well-defined. We need to show that for all $r \in R$, and 
    for all $x \in I$, we have $rx \in I$ and $xr \in I$. So we have $x + I = 0 + I = I$. Hence 
    \[
        rx + I = (r + I)(x + I) = (r + I)(0 + I) = 0 + I = I.
    \]
    Again we have $xr \in I$. Thus $I$ is an ideal in $R$.
\end{proof}

\begin{theorem}[Ideal test]
    A nonempty subset $A$ of a ring $R$ is an ideal of $R$ provided
    \begin{enumerate}
        \item $A \neq \varnothing$.
        \item $a - b \in A$ whenever $a, b \in A$.
        \item $ra$ and $ar$ are in $A$ for all $a \in A$ and $r \in R$.
    \end{enumerate}
\end{theorem}

\begin{example}
    For any ring $R$, $\{0 \}$ and $R$ are ideals of $R$. The ideal $\{ 0 \}$ is called the trivial ideal.
\end{example}

\begin{example}
    For any positive integer $n$, the set 
    \[
        n\mathbb{Z} = \{ 0, \pm n, \pm 2n, \pm 3n, \ldots \}
    \]
    is an ideal of $\mathbb{Z}$.
\end{example}
\begin{solution}
    We can show $n\mathbb{Z} \unlhd \mathbb{Z}$ using ideal test.
    \begin{enumerate}
        \item Since $0 \in \mathbb{Z}$, then $n \cdot 0 = 0 \in n\mathbb{Z} \neq \varnothing$.
        \item For all $a, b \in n\mathbb{Z}$, we let $a=nt_1$ and $b=nt_2$ for $t_1, t_2 \in \mathbb{Z}$. We have 
            \[
                a - b = nt_1 - nt_2 = n(t_1 - t_2) \in n\mathbb{Z} 
            \]
            since $t_1 - t_2$ is also an integer.
        
        \item Whenever $a \in A$ and $r \in R$, let $a = nt', \> t' \in \mathbb{Z}$. We have 
            \[
                ar = (nt')r = n(t'r) \in n\mathbb{Z}, \quad t'r \in \mathbb{Z}.
            \]
            and 
            \[
                ra = r(nt') = nrt' = n(rt') = n(t'r) = ar \in n\mathbb{Z}.
            \]
        
        Therefore $n\mathbb{Z} \unlhd \mathbb{Z}$.
    \end{enumerate}
\end{solution}

\begin{example}
    Let $R$ be a commutative ring with unity and let $a \in R$. The set 
    \[
        \langle \mathfrak{a} \rangle = \{ r\mathfrak{a} \> | \> r \in R \}
    \]
    is an ideal of $R$ called the \bred{principal ideal} generated by $\mathfrak{a}$.
\end{example}
\begin{solution}
    Using ideal test to check
    \begin{enumerate}
        \item Since $R$ is a ring, then $0_R \in R$ and so $0_R = 0_R \cdot \mathfrak{a} \in \langle \mathfrak{a} \rangle \neq \varnothing$.
        \item For all $b, c \in \langle \mathfrak{a} \rangle$, let $b = r_1 \mathfrak{a}$, $c = r_2 \mathfrak{a}$, where 
        $r_1, r_2 \in R$. Then 
        \[
            b - c = r_1 \mathfrak{a} - r_2 \mathfrak{a} = (r_1 - r_2) \mathfrak{a} \in \langle \mathfrak{a} \rangle.
        \]
        and $r_1 - r_2$ is in $R$.

        \item For all $\olsi{a} \in \langle \mathfrak{a} \rangle$, $r \in R$, we let $\olsi{a} = r' \mathfrak{a}$ and 
        \[
            \olsi{a}r = (r' \mathfrak{a}) r = \mathfrak{a}(r'r) \in \langle \mathfrak{a} \rangle.
        \]
        where $r'r \in R$.

        On the other hand, 
        \[
            r\olsi{a} = r(r' \mathfrak{a}) = \mathfrak{a}(rr') = \mathfrak{a}(r'r) = \olsi{a} r \in \langle \mathfrak{a} \rangle.
        \]

        Therefore $\langle \mathfrak{a} \rangle$ is a subring of $R$.
    \end{enumerate}
\end{solution}

\subsection{Quotient Rings}

\begin{theorem}
    Let $R$ be a ring and let $A$ be a subring of $R$. The set of cosets 
    \[
        R/A = \{ r+A \> | \> r \in A \}
    \]
    is a ring under the operations 
    \begin{itemize}
        \item $(s+A) + (t+A) = s + t + A$
        \item $(s+A)(t+A) = st + A$
    \end{itemize}
    if and only if $A$ is an ideal of $R$.
\end{theorem}
\begin{proof}
    Let $R$ to be a ring, and let $A \trianglelefteq R$. For all $s+A, t+A$ in $R/A$ we define 
    addition as 
    \[
        \boxplus (s+A, t+A) := (s+A) \boxplus (t + A) = s+t+A
    \]
    and multiplication as 
    \[
        \odot(s+A, t+A) := (s+A) \odot (t + A) = st+A.
    \]
    We want to show $(R/A, \boxplus, \odot)$ is a ring.

    \begin{enumerate}
        \item (Closureness) Suppose that $s + A = s' + A$ and $t+A = t' + A$ for all $s, s', t, t' \in \mathbb{R}$. First we 
        need to show 
        \[
            (s+t) + A = (s' + t')+A. 
        \]
        We are going to express $s, t$ in term of $s', t'$ respectively.
        \begin{align*}
            s + A = s' + A \> &\Rightarrow s - s' \in A\\
            &\Rightarrow s - s' = a_1 \in A\\
            &\Rightarrow \fbox{$s = a_1 + s'$}\>, \quad a_1 \in A. \label{eq:r2.31} \tag{{\color{red} $\heartsuit$}}
        \end{align*}
        and 
        \begin{align*}
            t + A = t' + A \> &\Rightarrow t - t' \in A\\
            &\Rightarrow t - t' = a_2 \in A\\
            &\Rightarrow \fbox{$t = a_2 + t'$}\>, \quad a_2 \in A. \label{eq:r2.32} \tag{{\color{gray} $\clubsuit $}}
        \end{align*}
    
        Summing up \eqref{eq:r2.31} together with \eqref{eq:r2.32} we have 
        \[
            s + t = (a_1 + s') + (a_2 + t') = a_1 + a_2 + s' + t', \quad a_1 + a_2 \in A.
        \]
        Subtracting $s'+t'$ on both side of the equation yields 
        \[
            s+t-(s'+t') = a_1 + a_2, \Longrightarrow s+t-(s'+t') \in A.
        \]
    
        We have shown $R/A$ closed under addition $\boxplus$, we continue to proof 
        \[
            st+A = s't' + A.
        \]
        Which is equivalent to show $R/A$ closed under the multiplication $\odot$. Applying 
        the results that we found from \eqref{eq:r2.31}, \eqref{eq:r2.32} we have 
        \[
            st -s't' = a_1 a_2 + a_1t' + s'a_2
        \]
        Is $a_1t'$ in $A$? Of course, since $a_1 \in A, t' \in R \Longrightarrow a_1t' \in A \triangleleft R$.
        So as $s'a_2 \in A \triangleleft R$.

        \item (Existence of additive identity) For all $s+A \in R/A$, there exists $e+A \in R/A$ such that 
        \begin{align*}
            (s + A) \boxplus (e+A) = s + A &\Rightarrow (s+e) + A = 0 + s + A\\
            &\Rightarrow s+e = s\\
            &\Rightarrow e = 0_A.
        \end{align*}
        Thus the additive identity is $0_A + A$.

        \item (Existence of additive inverse) For all $s+A \in R/A$, there exists $r+A \in R/A$ such that 
        \begin{align*}
            (s + A) \boxplus (r+A) = 0 + A &\Rightarrow (s+r) + A = 0 + A\\
            &\Rightarrow s+r = 0\\
            &\Rightarrow r =-s.
        \end{align*}
        Thus the additive inverse of $s+A$ is $-s+A$ in $R/A$.

        \item (Associativity of multiplication) For all $s+A, t+A, u+A$ in $R/A$, compute 
        \begin{align*}
            (s+A) \odot [(t+A) \odot (u+A)] &= (s+A) \odot (tu + A)\\
            &= s(tu) +A\\
            &= (st)u + A\\
            &= [(st)+A] \odot (u+A)\\
            &= [(s+A) \odot (t+A)] \odot (u+A).
        \end{align*}
        Associativity in $\odot$ holds.

        \item (Existence of unity) For all $s+A \in R/A$, there exists $e'+A \in R/A$ such that 
    \begin{align*}
        (s + A) \odot (e'+A) = s + A &\Rightarrow se'+A = s + A\\
        &\Rightarrow se' = s\\
        &\Rightarrow e' = 1_A \in R.
    \end{align*}
    The multiplicative identity is $1_A + A$ in $R/A$.

    \item (Existence of multiplicative inverse) For all $s+A \in R/A$, there exists $r+A \in R/A$ such that 
    \begin{align*}
        (s + A) \odot (r+A) = 1_A + A &\Rightarrow sr+A = 1_A + A\\
        &\Rightarrow sr = 1_A\\
        &\Rightarrow r = s^{-1}.
    \end{align*}
    The multiplicative inverse of $s+A$ is $s^{-1} + A$ in $R/A$, provided $s^{-1}$ exists in $R$.

    \item (Distributive Law)  For all $s+A, t+A, u+A$ in $R/A$, compute 
    \begin{align*}
        (s+A) \odot [(t+A) \boxplus (u+A)] &= (s+A) \odot [(t+u) + A]\\
        &= s(t+u) + A\\
        &= st + su + A\\
        &= (st + A) + (su + A)\\
        &= (st+A) \boxplus (su+A)\\
        &= (s+A) \odot (t+A) \boxplus (s+A) \odot (u+A)
    \end{align*}
    Distributive law holds in $(R/A, \boxplus, \odot)$.
    \end{enumerate}

    Therefore $(R/A, \boxplus, \odot)$ is a ring.
\end{proof}

\begin{example}
    $\mathbb{Z}/4\mathbb{Z} = \{ 4\mathbb{Z},\>  1 + 4\mathbb{Z},\>  2+4\mathbb{Z},\> 3+4\mathbb{Z} \}$
\end{example}
\begin{solution}
    The integers with multiple of $4$ is 
    \[
        4\mathbb{Z} = \{\ldots, -8, -4, 0, 4, 8, 12, \ldots \}.
    \]
    The left ideals are 
    \begin{align*}
        1+_4 4\mathbb{Z} &= \{ \ldots, -7, -3, 1, 5, 9, 13, \ldots \}\\
        2+_4 4\mathbb{Z} &= \{ \ldots, -6, -2, 2, 6, 10, 14, \ldots \}\\
        3+_4 4\mathbb{Z} &= \{ \ldots, -5, -1, 3, 7, 11, 15, \ldots \}\\
        4+_4 4\mathbb{Z} &= \{ \ldots, -4, 0, 4, 8, 12, 16, \ldots \} = 4\mathbb{Z}
    \end{align*}
\end{solution}

\begin{definition}[characteristic of ring]
    The characteristic of a ring $R$ is the least positive integer $n$ such that $nx = 0$ for all $x \in R$. If no such integer exists, we say that 
    $R$ has characteristic 0. The characteristic of $R$ is denoted by $\text{char}(R)$.
\end{definition}

\begin{example}
    The characteristic of ring $\mathbb{Z}_7$ is 7.
\end{example}

\begin{example}
    Find the characteristic of $R = \mathbb{Z}[i] / \langle 2 - i \rangle$.
\end{example}
\begin{solution}
    Given that 
    \[
        R = \mathbb{Z}[i] / \langle 2 - i \rangle = \{ a + bi + \langle 2-i \rangle \}.
    \]
    Observe that 
    \[
        2 - i + \langle 2-i \rangle = 0 + \langle 2-i \rangle 
    \]
    and so we treat $2 - i$ as zero, and $i = 2$ and so 
    \[
        -1 = i^2 = 4 \Longrightarrow 5 = 0 \in R.
    \]
    Thus $R = \{ 0+\langle 2-i \rangle, 1 + \langle 2-i \rangle, 2 + \langle 2-i \rangle, 
    3 + \langle 2-i \rangle, 4 + \langle 2-i \rangle\}$. So $\text{Char}(R) = 5$.
\end{solution}

\begin{definition}[Prime ideal]
    An ideal $I$ in a commutative ring $R$ is said to be prime if $I \neq R$ and whenever 
    $ab \in I$, then either $a \in I$ or $b \in I$.
\end{definition}

\begin{lemma}
    Let $R$ be a commutative ring with unity, and $I$ be an ideal in $R$. Then $I$ is 
    a prime ideal in $R$ if and only if $R/I$ is an integral domain.
\end{lemma}
\begin{proof}
    $R/I$ will therefore be an integral domain and only if it has no zero divisors. This condition 
    is equivalent to the condition that 
    \[
        (a+I)(b+I) = I \iff a + I = I \> \textsf{ or } \> b + I = I.
    \]
    Thus $R/I$ is an integral domain if and only if $ab + I = I$ implies that 
    $a + I = I$ or $b+I = I$ or, in other words, if and only if 
    $ab \in I$ implies that $a \in I$ or $b \in I$, which is to say that $I$ is a prime ideal 
    in $R$. 
\end{proof}

\begin{definition}[Maximal ideal]
    An ideal $I$ in a ring $R$ is said to be maximal if $I \neq R$ and whenever $J$ is an ideal such that
    \[
        I \subset J \subset R
    \]
    then $I = J$ or $J = R$.
\end{definition}

\begin{lemma}
    Consider $R$ is a ring with nonzero unity, and $M$ is an ideal such that $M \neq R$. If $R/M$ is a 
    division ring, then $M$ is a maximal ideal.
\end{lemma}
\begin{proof}
    Suppose $I$ is an ideal such that $M \subsetneq I \subseteq R$. Then $\exists a \in I \> s.t. \> a \notin M$. 
    Then $a + M \neq 0 + M$ and there exists $b + M \in R/M$ such that 
    \[
        (a+M)(b+M) = 1_R + M \Longrightarrow (1_R - ab) \in M \Longrightarrow ab+m = 1_R
    \]
    for some $m \in M$. Since $ab \in I$ and $m \in M \subset I$. Also $1_R \in I \Longrightarrow I = R$. Thus $M$ is 
    a maximal ideal.
\end{proof}

\begin{theorem}
    Let $M$ be an ideal in a commutative ring $R$ with identity. Then $M$ is a maximal ideal if and only if
    the quotient ring $R/M$ is a field. 
\end{theorem}
\begin{proof}
    $(\Leftarrow)$ If $R/M$ is a field, then $M$ is a maximal ideal by previous lemma.

    $(\Rightarrow)$ Since $M \neq R$, $R/I$ is a commutative ring with 
    $1_R + R \neq 0_R + M$. Take any nonzero $a + M \in R/M, a \notin M$ and put 
    \[
        N := Ra + M = \{ ra+m \> | \> r \in R, m \in M \}.
    \]
    Note that $Ra$ is an ideal and $M$ is also an ideal ($Ra = \langle a \rangle$). 
    Thus $Ra + M$ is ideal that include $M$.

    Since $M$ is maximal, this implies that $N = R \Longrightarrow 1_R \in N$. 
    $ra + m = 1_R$ for some $r \in R, m \in M$. Compute 
    \begin{align*}
        ra + m = 1_R &\Rightarrow ra + M = 1_R + M & \text{Since } (ra-1_R) \in M\\
        &\Rightarrow (a+M)(r+M) = 1_R + M.
    \end{align*}
    We can now see that $a+M$ is actually a unit in $R/M$. Hence $R/M$ is a field.
\end{proof}

\begin{corollary}
    In a commutative ring $R$ with unity, every maximal ideal is a prime ideal.
\end{corollary}
\begin{proof}
    If $I$ is a maximal ideal in $R$, then $R/I$ is a field. Every field is an integral domain, so $R/I$ is 
    also an integral domain, and $I$ is a prime ideal.
\end{proof}

\begin{example}
    The ideal $\langle x^2 + 1 \rangle$ is maximal in $\mathbb{R}[x]$.
\end{example}
\begin{solution}
    Suppose that $A$ is an ideal of $\mathbb{R}[x]$ that properly 
    contains $\langle x^2 + 1 \rangle$. We will 
    prove that $A = \mathbb{R}[x]$ by showing that $A$ contains some nonzero real number 
    $c$. Then 
    \[
        1 = \frac{1}{c} c \in A
    \]
    and therefore $A = \mathbb{R}[x]$.

    Let a polynomial $f(x) \in A$, but $f(x) \notin \langle x^2 + 1 \rangle$. Then 
    \[
        f(x) = q(x) (x^2 + 1) + r(x),
    \]
    where $r(x) = ax + b$, where $a,b$ both are nonzero, and hence 
    \[
        a^2 x^2 - b^2 = (ax+b)(ax-b) \in A \textsf{ and } a^2(x^2 + 1) \in A.
    \] 
    So that 
    \[
        0 \neq a^2 + b^2 = (a^2x^2 + a^2) - (a^2x^2 - b^2) \in A.
    \]
\end{solution}

\begin{example}
    The ideal $\langle x^2 + 1 \rangle$ is not prime in $\mathbb{Z}_2[x]$.
\end{example}
\begin{solution}
    (Counterexample) Take 
    \[
        (x+1)^2 = x^2 + 2x + 1 = x^2 + 1 \in \langle x^2 + 1 \rangle
    \]
    but $1 + x \notin \langle x^2 + 1 \rangle$.
\end{solution}

\section{Ring homomorphism}

\begin{definition}[Ring homomorphism]
    A ring homomorphism $f$ from a ring $(R, {\color{red} \oplus}, {\color{red} \odot})$ to a ring 
    $(S, {\color{blue} \boxplus}, {\color{blue} \boxdot})$ is a mapping from $R$ to $S$ that preserves the 
    ring additions (${\color{red} \oplus}, {\color{blue} \boxplus}$) and multiplications (${\color{red} \odot}, {\color{blue} \boxdot}$); that is, 
    \begin{equation*}
        f(a \> {\color{red} \oplus} \> b) = f(a) \> {\color{blue} \boxplus} \> f(b) 
    \end{equation*}
    and 
    \begin{equation*}
        f(a \> {\color{red} \odot} \> b) = f(a) \> {\color{blue} \boxdot} \> f(b) 
    \end{equation*}
    A ring homomorphism that is one-to-one and onto is called the \bred{ring isomorphism}.
\end{definition}

\begin{example}
    The map $\phi: \mathbb{Z} \to \mathbb{Z}_3$ defined by 
    \[
        \phi(x) = x \> (\text{mod }3) \quad \forall x \in \mathbb{Z}
    \]
    is a ring homomorphism.
\end{example}
\begin{solution}
    Clearly, for all $x, y \in \mathbb{Z}$
    \begin{align*}
        \phi(x + y) = (x + y) \> (\text{mod }3) &= (x \> \text{mod }3) + (y \> (\text{mod }3))\\
        &= \phi(x) +_3 \phi(y)
    \end{align*}

    and 
    \begin{align*}
        \phi(xy) = (xy) \> (\text{mod }3) &= (x \> \text{mod }3) \cdot (y \> (\text{mod }3))\\
        &= \phi(x) \cdot_3 \phi(y)
    \end{align*}

    This is an example of a map that respects both operations.
\end{solution}

\begin{example}
    Consider the map $\phi: \mathbb{Z}_4 \to \mathbb{Z}_6, \phi(x) = 3x$ for all $x$ in 
    $\mathbb{Z}_4$. $\phi$ is a ring homomorphism.
\end{example}
\begin{solution}
    For all $x, y \in \mathbb{Z}$, we check that
    \begin{align*}
        \phi(x + y) = 3(x + y) \> (\text{mod }6) &= (3x \> \text{mod }6) + (3y \> (\text{mod }6))\\
        &= \phi(x) +_6 \phi(y)
    \end{align*}

    and 
    \begin{align*}
        \phi(xy) = 3(xy) \> (\text{mod }6) = 9(xy) \> (\text{mod }6) &= (3x \> \text{mod }6) \cdot (3y \> (\text{mod }6))\\
        &= \phi(x) \cdot_6 \phi(y)
    \end{align*}

    this map preserves both operations. So $\phi$ is a ring homomorphism.

    In our calculation, we can have used the fact that $3 = 9 \> (\text{mod }6)$. The jump from 3 to 9 modulo 6 
    can be better seem in 
    \[
        3 \> (\text{mod }6) = \phi(1) = \phi(1 \cdot 1) = \phi(1)\, \phi(1) = 3 \cdot 3 = 9 \> (\text{mod }6).
    \]
\end{solution}

\begin{example}
    For $a, b \in \mathbb{R}$, let $A(a,b) = M_2(\mathbb{R})$ be defined by 
    \[
        A(a, b) = \begin{bmatrix}
            a & b \\ -b & a 
        \end{bmatrix}.
    \]
    Let $R = \{ A(a,b) \> | \> a, b \in \mathbb{R} \} \subseteq M_2(\mathbb{R})$. Then $R \cong \mathbb{C}$.
\end{example}
\begin{solution}
    Let $\phi: R \to \mathbb{C}$ be defined by 
    \[
        \phi(A(a,b)) = a + bi \in \mathbb{C}.
    \]
    We show firstly that $\phi$ is a ring homomorphism. 

    For addition, we have 
    \begin{align*}
        \phi(A(a,b) + A(c,d)) &= \phi \left( A(a+c, b+ d)\right) \\
        &= (a + c) + (b+d)i\\
        &= (a+bi) + (c + di)\\
        &= \phi\left( A(a,b) \right) + \phi\left( A(c,d) \right).
    \end{align*}

    For multiplication, we have 
    \begin{align*}
        \phi(A(a,b) + A(c,d)) &= \phi \left( \begin{bmatrix}
            a & b \\ -b & a
        \end{bmatrix}
        \begin{bmatrix}
            c & d \\ -d & c
        \end{bmatrix} \right)  \\
        &= \phi \left( \begin{bmatrix}
            ac-bd & ad + bc \\ -(ad+bc) & ac - bd
        \end{bmatrix} \right)\\
        &= \phi \left( A(ac-bd, ad + bc )\right)\\
        &= (a+bi)(c+di)\\
        &= \phi \left( A(a,b)\right)\, \phi \left( A(c,d)\right).
    \end{align*}

    Now, $\phi$ is one-to-one and onto since $\phi \left( A(a,b) \right) = a+bi = 0$ if and only if 
    $a = b = 0$, and $Ker\, \phi =\{ A(0,0) \}$ is trivial.
\end{solution}

\begin{example}
    Show that the equation $2x^3 - 5x^2 + 7x - 8 = 0$ has no solutions in $\mathbb{Z}$.
\end{example}
\begin{solution}
    Let $\phi: \mathbb{Z} \to \mathbb{Z}_3$ be the natural homomorphism $\phi(x) = x \text{ mod } 3$. Suppose that there 
    is an integer $a \in \mathbb{Z}$ such that
    \[
        2a^3 - 5a^2 + 7a - 8 = 0.
    \]
    Then 
    \[
        0 = \phi(0) = \phi(2a^3 - 5a^2 + 7a - 8) = 2\phi(a)^3 - 5\phi(a)^2 + 7\phi(a) - 8.
    \]
    Since $-5 = 7 = -8 = 1 \, (\text{mod }3)$ in $\mathbb{Z}_3$, we have 
    \[
        2\phi(a)^3 - 5\phi(a)^2 + 7\phi(a) - 8 \> = \> 2\phi(a)^3 + \phi(a)^2 + \phi(a) + 1
    \]
    and thus $2b^3 + b^2 + b + 1 = 0$, where $b = \phi(a)$ in $\mathbb{Z}_3$.

    However, one can easily check that no element $b \in \{ 0, 1, 2 \}$ in $\mathbb{Z}_3$ is a solution to this equation. 
    Therefore there is no such integer $a \in \mathbb{Z}$ to the original equation.
\end{solution}

\begin{example}[Tutorial]
    Show that the rings $2 \mathbb{Z}$ and $3 \mathbb{Z}$ are not isomorphic.
\end{example}
\begin{solution}
    Assume the contrary and let $\phi: 2 \mathbb{Z} \to 3\mathbb{Z}$ to be an isomorphism. Let us examine 
    $\phi(2)$. Note that for some $k \in \mathbb{Z}, \phi(2) = 3k$. Since $\phi$ is a homomorphism, 
    \[
        \phi(k) = \phi(2 + 2) = \phi(2) + \phi(2) = 3k + 3k = 6k.
    \]
    But $\phi$ is a ring homomorphism and 
    \[
        \phi(k) = \phi(2 \cdot 2) = \phi(2)\, \phi(2) = (3k)(3k) = 9k^2.
    \]
    This implies that $6k = 9k^2 \Longrightarrow k = 0$ or $k = \frac{2}{3}$.

    For $k = 0 \Longrightarrow \phi(x) = 0$ is not one-to-one and not onto. Also, $k = \frac{2}{3} \notin \mathbb{Z}$, and thus 
    $\phi$ cannot be an isomorphism.
\end{solution}

\begin{example}
    Determine all ring homomorphism from $\mathbb{Z}$ to $\mathbb{Z}_6$.
\end{example}
\begin{solution}
    Since $\mathbb{Z}$ is generated from $1$ by addition and subtraction, if a ring homomorphism $f: \mathbb{Z} \to \mathbb{Z}_6$, then for any 
    $a \in \mathbb{Z}$, we have 
    \[
        f(a) = am
    \]
    where $m = f(1)$. Then $f$ is linear, so 
    \[
        f(a) + f(b) = am + bm = (a+b)m = f(a+b) \quad \forall a, b \in \mathbb{Z}.
    \]
    So $f$ is a ring homomorphism if and only if 
    \[
        0 = f(ab) - f(a)f(b) = abm - (am)(bm) = ab(m - m^2)
    \]
    for any $a,b \in \mathbb{Z}$. 

    In particular, taking $a = b = 1$, we need to find $m$ such that $0 = m - m^2 \> (\text{mod} 6)$. Working modulo 6 one by one 
    \begin{align*}
        0 - 0^2 = 0-0=0, &1-1^2=1-1=0, &2 - 2^2 = 2 - 4 = 4 \neq 0\\
        3 - 3^2 = 3 - 9 = -6 = 0, &4-4^2 = 4 -16 = 0, &5 - 5^2 = 5 - 25 = 2\neq 0
    \end{align*}
    
    The possible values of $m$ are $0, 1, 3$ and $4$. So the homomorphisms are as follow
    \begin{itemize}
        \item $f(a) = 0 \> (\text{mod} 6), \quad \forall a \in \mathbb{Z}$.
        \item $f(a) = a \> (\text{mod} 6), \quad \forall a \in \mathbb{Z}$.
        \item $f(a) = 3a \> (\text{mod} 6), \quad \forall a \in \mathbb{Z}$.
        \item $f(a) = 4a \> (\text{mod} 6), \quad \forall a \in \mathbb{Z}$.
    \end{itemize}
\end{solution}

\begin{theorem}[The first isomorphism for rings]
    Let $f$ be a ring homomorphism from ring $R$ ro $S$. Then the mapping from $R / ker(f)$ to $f(R)$, given by 
    \[
        r + ker(f) \to f(r)
    \]
    is an isomorphism. In symbols, $R/ker(f) \cong f(R)$.
\end{theorem}
\begin{proof}
    Define a map $f: R/K \to \Ima \olsi{f}$ by 
    \[
        f(a + K) = \olsi{f}(a)\quad \forall a \in R, a+ K \in R/K.
    \]
    \begin{enumerate}
        \item Since $f$ is well defined, so 
        \begin{align*}
            a + K = b + K &\Rightarrow a - b \in K\\
            &\Rightarrow \olsi{f}(a - b) = 0_S\\
            &\Rightarrow \olsi{f}(a) - \olsi{f}(b) = 0_S\\
            &\Rightarrow \olsi{f}(a) = \olsi{f}(b)
        \end{align*}

        \item $f$ is injective since 
            \begin{align*}
                f(a +K) = f(b+K) &\Rightarrow \olsi{f}(a) = \olsi{f}(b)\\
                &\Rightarrow \olsi{f}(a-b) = 0_S\\
                &\Rightarrow a - b \in K\\
                &\Rightarrow a + K = b + K.
            \end{align*}
        
        \item $f$ is surjective. For all $f(a) \in \Ima \olsi{f}, \exists a+K \in R/K$
            such that $\olsi{f}(a+K) = f(a)$.
        
        \item $f$ is homomorphism, 
            \begin{align*}
                f(a + K + b + K) &\Rightarrow f((a+b) + K)\\
                &\Rightarrow \olsi{f}(a+b)\\
                &\Rightarrow \olsi{f}(a) + \olsi{f}(b)\\
                &\Rightarrow f(a+K) + f(b+K).
            \end{align*}

            \begin{align*}
                f(a + K) \cdot f(b + K) &\Rightarrow f(ab + K)\\
                &\Rightarrow \olsi{f}(ab)\\
                &\Rightarrow \olsi{f}(a) \cdot \olsi{f}(b)\\
                &\Rightarrow f(a+K) \cdot f(b+K).
            \end{align*}
    \end{enumerate}
    Thus $f: R/K \cong \Ima f$ as rings.
\end{proof}

\begin{example}
    Let $\phi: \mathbb{Z} \times \mathbb{Z} \to \mathbb{Z}_3$ be the ring homomorphism defined by 
    \[
        \phi((a,b)) = b \text{ mod }3.
    \]
    Then $ker(\phi) = \mathbb{Z} \times 3 \mathbb{Z}$ and $(\mathbb{Z} \times \mathbb{Z}) / (\mathbb{Z} \times 3 \mathbb{Z})$ 
    is isomorphic to $\mathbb{Z}_3$, which is a field. Thus $\mathbb{Z} \times 3\mathbb{Z}$ is a maximal ideal of 
    $\mathbb{Z} \times \mathbb{Z}$.
\end{example}

\section{Polynomial rings}

\begin{definition}
    Let $R$ be a commutative ring. We define 
    \begin{equation}
        R[x] = \{ r_nx^n + r_{n-1}x^{n-1} + \cdots + r_1 x + r_0 \> | \> r_i \in R \}.
    \end{equation}
\end{definition}

The letter $x$ here can be thought of a variable or just a placeholder. Either way the familiar structure 
allows us to add, subtract and multiply these as we do traditional polynomials even if the ring 
were some strange abstract entity.

\section{Factorization of polynomials}

\begin{theorem}[Division algorithm]
    Let $R$ be a ring with identity and $f(x),g(x) \in R[x]$ with 
    $g(x) \neq 0$. Then there exists unique polynomials 
    $q(x)$ and $r(x)$ in $R[x]$ such that 
    \begin{equation}
        f(x) = q(x) g(x) + r(x)
    \end{equation}
    and $\text{deg}(r) < \text{deg}(g)$. $r(x) = 0$ if there is no remainder.
\end{theorem}

\begin{proof}
    The basic idea is to formalize the process of long division in an inductive sense. We omit the details here. 
    They're boring here.
\end{proof}

\begin{example}
    In $\mathbb{Z}_3$ we can divide $2x^2 + 1$ into $x^4 + 2x^3 + 2x + 1$. Then we have 
    \[
        x^4 + 2x^3 + 2x + 1 = (2x^2 + 1)(2x^2 + x + 2)
    \]
\end{example}

\begin{theorem}[Factor theorem]
    Let $F$ be a field, $a \in F$ and $f(x) \in F[x]$. Then $a$ is a \bgreen{root} (or \bgreen{zero}) of 
    $f(x)$ if and only if $x - a$ is a factor of $f(x)$.
\end{theorem}
\begin{proof}
    $(\Rightarrow)$ Assume that $a \in F$ is a zero of $f(x) \in F[x]$. We wish to show that 
    $x - a$ is a factor of $f(x)$. To do so, apply the division algorithm. By division algorithm, 
    $\exists \,$ unique polynomials $q(x)$ and $r(x)$ such that 
    \[
        f(x) = (x-a)q(x) + r(x)
    \]
    and the $deg(r) < deg(x-a) = 1$, so $r(x) = c \in F$, where $c$ is a constant. Also, the fact that 
    $a$ is a zero of $f(x)$ implies $f(a) = 0$. So 
    \[
        f(x) = (x-a)q(x) + c \> \Longrightarrow 0 = f(a) = (a-a)q(a) + c.
    \]
    Thus $c = 0$, and $x - a$ is a factor of $f(x)$.

    $(\Leftarrow)$ On the other way, we want to show 
\end{proof}

\begin{definition}[Algebraically closed]
    Given $F$ a field, we call $F$ \bred{algebraically closed} if every $f \in F[x]$ such that 
    $deg(F) > 0$ has a root in $F$.
\end{definition}

\begin{example}
    Show that $x^2 + 3x - 4 \in \mathbb{Z}_{12}[x]$ has 4 roots.
\end{example}
\begin{solution}
    We list down all the values of $f(x) = x^2 + 3x - 4$ for $x = 0, 1, \ldots, 11$.

    \begin{center}
        \begin{tabular}{|c|c|c|c|c|c|c|c|c|c|c|c|c|}
            \hline
            $x$ & 0 & 1 & 2 & 3 & 4 & 5 & 6 & 7 & 8 & 9 & 10 & 11\\
            \hline
            $x^2+3x-4\> (\text{mod } 12)$ & 8 & {\color{red} 0} & 6 & 2 & {\color{red} 0} & {\color{red} 0} & 
            2 & 6 & {\color{red} 0} & 8 & 6 & 6\\
            \hline
        \end{tabular}
    \end{center}

    which now we can see: $x^2 + 3x - 4$ has 4 zeros in $\mathbb{Z}_{12}[x]$. Thus, a polynomial of degree $n$ can have more than $n$ roots 
    in a ring. The problem is that $\mathbb{Z}_{12}$ is not a domain: $(x+4)(x-1) = 0$ 
    does not imply one of the factors must be zero.
\end{solution}

\begin{example}
    Show that the polynomial $2x^3 + 3x^2 -7x - 5$ can be factored into linear factors in $Z_{11}[x]$.
\end{example}
\begin{solution}
    We can use synthetic division,

    \begin{center}
        \begin{tabular}{cccc|c}
            $2$ & $3 = -8$ & $-7=4$ & $6$ & \\
            & $-2$ & $-10$ & $-6$ & $-1$\\
            \cmidrule{1-4}
            $2$ & $-10 = 1$ & $-6$ & & \\
            & $-4$ & $6$ & & $-2$\\
            \cmidrule{1-4}
            $2$ & $-3$ &  & & \\
        \end{tabular}
    \end{center}

    Thus, $2x^3 + 3x^2 -7x - 5 = (x+1)(x+2)(2x-3)$ in $\mathbb{Z}_{11}[x]$.
\end{solution}

\subsection{Irreducibility tests}

There are various methods to check if a polynomial in $\mathbb{Z}[x]$ is irreducible in $\mathbb{Q}[x]$.

\begin{theorem}[Rational root test]
    Let 
    \[ f(x) = a_nx^n + a_{n-1}x^{n-1} + \cdots + a_1x + a_0 \label{eq:r2.2} \tag{{\color{orange} $\bigstar $}} \]
    be a polynomial with integers coefficients. If $r \neq 0$ and 
    the rational number $r/s$ (in lowest terms) is a root of $f(x)$, then 
    $r|a_0$ and $s|a_n$.
\end{theorem}
\begin{proof}
    Plug $x = r/s$ into \eqref{eq:r2.2} and equating with zero. The equation is now 
    \[
        a_n \left( \frac{r}{s}\right)^n + a_{n-1}\left( \frac{r}{s}\right)^{n-1} + \cdots + a_1\left( \frac{r}{s}\right) + a_0
        = 0.
    \]
    Again multiplying $s^n$ on both sides
    \[
        a_nr^n + a_{n-1}r^{n-1}s + \cdots + a_1rs^{n-1}x + a_0s^n = 0.
    \]
    Factoring $r$ out and moving $a_0s^n$ to the right-hand side. We obtained 
    \[
        r(a_nr^{n-1} + a_{n-1}r^{n-2}s + \cdots + a_1s^{n-1}x) = -a_0s^n.
    \]
    Since $\gcd (r,s) = 1$, thus $r|a_0$ and similarly $s|a_n$.
\end{proof}

\begin{example}
    The polynomial $f(x) = 2x^4 + x^3 - 21x^2 - 14x + 12$ is reducible in $\mathbb{Q}[x]$.
\end{example}
\begin{solution}
    If $r/s$ is a root of $f(x)$, where $r|12$ and $s|2$. Thus the possible roots are 
    \[
        \pm 1,\> \pm 2,\> \pm 3,\> \pm 4,\> \pm 6,\> \pm 12,\> \pm \frac{1}{2},\> \pm \frac{3}{2}.
    \]
    In fact, $f(x) = (x+3) \left(x - \frac{1}{2} \right)(2x^2 - 4x - 8) \in \mathbb{Q}[x]$.
\end{solution}

\begin{example}
    The polynomial $g(x) = x^3 + 4x^2 + x - 1$ is irreducible in $\mathbb{Q}[x]$.
\end{example}
\begin{solution}
    The possible roots are $\{-1, 1\}$. However
    \[
        g(1) = 1 + 4 + 1 - 1 = 5 \quad \text{and }\quad g(-1) = -1 + 4 - 1 - 1 = 1
    \]
    So $g(x)$ has no root and $\deg g(x) = 3$. Thus $g(x)$ is irreducible over $\mathbb{Q}[x]$.
\end{solution}

\begin{theorem}[Mod $p$ Irreducibility test]
    Let $p$ be a prime and let $f(x) \in \mathbb{Z}[x]$ with degree 1 or greater. Let 
    $\overline{f} \in \mathbb{Z}_p[x]$ obtained by reducing all of $f(x)$'s coefficients 
    mod $p$. Then if 
    \begin{equation}
        \deg(\overline{f}) = \deg(f)
    \end{equation}
    and $\overline{f}$ is irreducible over $\mathbb{Z}_p$ then $f(x)$ is irreducible over 
    $\mathbb{Q}$.
\end{theorem}
\begin{proof}
    Assume that $f(x) = p(x) q(x)$ in $\mathbb{Z}[x]$. Since $\phi: \mathbb{Z}[x] \to \mathbb{Z}_p[x]$
    defined by $\phi f(x) = \overline{f}(x)$ is a ring homomorphism. So 
    \[
        \overline{f}(x) = \overline{p(x) q(x)} = \overline{p}(x) \overline{q}(x).
    \]
    If $p \nmid a_k$, then $p$ does not divide the leading coefficients of $p(x)$ and $q(x)$. Thus 
    $\deg \overline{p}(x) = \deg p(x)$ and $\deg \overline{q}(x) = \deg q(x)$.
\end{proof}

\begin{example}
    The polynomial $f(x) = x^5 + 8x^4 + 3x^2 + 4x + 7$ is irreducible in $\mathbb{Q}$.
\end{example}
\begin{solution}
    We define
    \[
        \olsi{f}(x) = x^5 + x^2 + 1 \in \mathbb{Z}_2[x].
    \]
    By rational root test, the only possible root is $0.1$ from $\mathbb{R}$ but it is not an inetger.
    There are several quadratic polynomials in $\mathbb{Z}_2[x]$ such as 
    \[
        x^2 + x + 1, \quad x^2 + 1,\quad x^2 + x, \quad x^2.
    \]
    Since $x^2 + 1, x^2 + x, x^2$ both have roots, they cannot be factor of $\olsi{f}$. The only possible 
    factor of $\olsi{f}$ is $x^2 + x + 1$. Thus 
    \[
        x^5 + x^2 + 1 = (x^2 + x+ 1)(x^3 + ax^2 + bx + c).
    \]
    Equating coefficients of both sides, we have 
    \[
        \begin{cases}
            1 + a = 0\\
            1 + a + b = 0\\
            a + b + c = 0\\
            b + c = 0\\
            c = 1
        \end{cases}.
    \]
    On solving yields $a = -1 = 1 (\text{mod } 2)$,\> $b = 0$ and $c = 1$ but $b + c \neq 0$ and is contradiction.
    So $f(x)$ does not has a quadratic factor. It means that $f(x)$ is irreducible in both 
    $\mathbb{Z}_2[x]$ and $\mathbb{Z}$. So $f(x)$ is also irreducible in $\mathbb{Q}[x]$. 
\end{solution}

\begin{theorem}[Eisenstein's criterion]
    Let $f(x) = a_0 + a_1x + a_2x^2 + \cdots + a_nx^n \in \mathbb{Z}[x] \setminus \{ 0 \}$. 
    If there is a prime number $p$ such that 
    $p \nmid a_n$, but $p | a_{n-1}, \ldots p | a_{2}$ and $p^2 | a_0$. Then $f(x)$ is 
    irreducible over $\mathbb{Q}$.
\end{theorem}
\begin{proof}
    Suppose that $f(x)$ is reducible over $\mathbb{Q}$ then 
    \[
        f(x) = g(x) h(x)
    \] 
    and $g(x), h(x)$ are nonconstant polynomials.

    Let 
    \[
        f(x) = a_nx^n + a_{n-1}x^{n-1} + \cdots + a_1x + a_0,
    \]
    \[
        g(x) = b_rx^r + b_{r-1}x^{r-1} + \cdots + b_1x + b_0,
    \]
    \[
        h(x) = c_sx^s + c_{s-1}x^{s-1} + \cdots + c_1x + c_0.
    \]
    Since $p|a_0 =b_0c_0 \Longrightarrow p | b_0$ or $p|c_0$, and $p^2 \nmid a_0$. This implies that 
    $p$ divides only one of them. Assume that $p | b_0$ and $p \nmid c_0$, then 
    \[
        p|a_0 = b_0c_1 + b_1c_0.
    \]
    Since $p|b_0c_1$ and $p \nmid c_0 \Longrightarrow p|b_1$. Assume that $p|b_i \> \forall 0 \leq i < m$ 
    for some $m \leq r$. Then 
    \[
        p|a_m = \sum_{\substack{i+j=m \\ j \leq s}} b_ic_j \Longrightarrow p|b_mc_0 \Longrightarrow p|b_m.
    \]
    By mathematical induction, $p|b_r$. Thus $p|a_n = b_rc_s$. This contradicting the fact that $f(x)$ 
    is reducible.
\end{proof}

\begin{example}
    $x^9 + 5$ is irreducible in $\mathbb{Q}[x]$ with $p=5$.
\end{example}

\begin{example}
    $x^{17} + 6x^{13} - 15x^{4} + 3x^2 - 9x + 12$ is irreducible in $\mathbb{Q}[x]$ with $p=3$.
\end{example}

\begin{example}
    $x^n + 5$ is irreducible in $\mathbb{Q}[x]$ for all $n \geq 1$. There are irreducible polynomials of 
    every degree in $\mathbb{Q}[x]$.
\end{example}

\begin{corollary}
    For any prime $p$, the $p$-th cyclotomic polynomial
    \[
        \Phi_p(x) = x^{p-1} + x^{p-2} + \cdots + x + 1
    \]
    is irreducible over $\mathbb{Q}$.
\end{corollary}
\begin{proof}
    Let $\zeta = e^{2\pi i / n}$. Then $\zeta, \zeta^2, \ldots, \zeta^n$ are the $n$-th roots of unity. They form 
    the vertices of a regular $n$-gon in the complex plane. If $\gcd(a,n) > 1$ then $\zeta^a$ is a root of unity 
    of order $n / \gcd(a,n) < n$, but if $\gcd(a,n) = 1$ then $\zeta$ is not a root of lower order, and in this case we call 
    $\zeta^a$ a primitive $n$-th root of unity. We define the $n$-th cyclotomic polynomial $\Phi_n(x)$ to be the monic polynomial
    of degree $\phi(n)$ whose roots are the primitive $n$-th root of unity: 
    \begin{equation}
        \Phi_n(x) = \prod_{\substack{a = 1\\ \gcd(a,n)=1}}^{n}(x - \zeta^a).
    \end{equation}

    The first few cyclotomic polynomials are as follows:

    \newpage
    \begin{center}
        \begin{tabular}{c|c}
            $n = $ & \\
            \hline
                $1$ & $\Phi_1(x) = x-1$\\[0.225em]
                $2$ & $\Phi_2(x) = x + 1$\\[0.225em]
                $3$ & $\Phi_3(x) = x^2 + x + 1$\\[0.225em]
                $4$ & $\Phi_4(x) = x^2 + 1$\\[0.225em]
                $5$ & $\Phi_5(x) = x^4 + x^3 + x^2 + x + 1$\\[0.225em]
                $6$ & $\Phi_6(x) = x^2 - x + 1$\\[0.225em]
                $7$ & $\Phi_7(x) = x^6 + x^5 + x^4 + x^3 + x^2 + x + 1$\\[0.225em]
                $8$ & $\Phi_8(x) = x^4 + 1$\\[0.225em]
                $9$ & $\Phi_9(x) = x^6 + x^3 + 1$\\[0.225em]
                $10$ & $\Phi_{10}(x) = x^4 - x^3 + x^2 - x + 1$\\[0.225em]
                $11$ & $\Phi_{11}(x) = x^{10} + x^9 + x^8 + x^7 + x^6 + x^5 + x^4 + x^3 + x^2 + x + 1$\\[0.225em]
                $12$ & $\Phi_{12}(x) = x^4 - x^2 + 1$\\[0.225em]
                $13$ & $\Phi_{13}(x) = x^{12} + x^{11} + x^{10} + x^9 + x^8 + x^7 + x^6 + x^5 + x^4 + x^3 + x^2 + x + 1$\\[0.225em]
                $14$ & $\Phi_{14}(x) = x^6 - x^5 + x^4 - x^3 + x^2 -x + 1$\\[0.225em]
                $15$ & $\Phi_{15}(x) = x^8 - x^7 + x^5 - x^4 + x^3 -x + 1$\\[0.225em]
                $16$ & $\Phi_{16}(x) = x^8 + 1$\\[0.225em]
        \end{tabular}
    \end{center}

    Let $p$ denote a given prime number. For any polynomial $f(x)$ with integral coefficients let 
    $\overline{f}(x)$ be the polynomial whose coefficients are the residue classes (mod $p$) determined by the coefficients
    of $f(x)$. Thus the assertion $\overline{f} = \overline{g}$ means that there is a polynomial $h(x)$ with 
    integral coefficients such that $f(x) = ph(x)$.

    \begin{lemma}
        (Sch\"{o}nemann, 1846) Let $A(x)$ be a monic polynomial with integral coefficients, for instance
        \[
            A(x) = x^n + a_{n-1}x^{n-1} + \cdots + a_0 = \prod^{n}_{i = 1} (x - \alpha_i).
        \]
        Let $p$ be a prime, and put 
        \[
            C(x) = \prod_{i=1}^{n} (x - \alpha^p_i).
        \]
        Then $\overline{C} = \overline{A}$.
    \end{lemma}
    \begin{proof}
        Let $\sigma_k(\alpha)$ denote the $k$-th symmetric function of $\alpha_i$. When $\sigma_k(\alpha)^p$ is expanded 
        by the multinomial theorem, all coefficients except the extreme ones are divisible by $p$. That is,
        \[
            \frac{\sigma_k(\alpha_1, \alpha_2, \ldots, \alpha_n)^p - \sigma_k(\alpha^p_1, \alpha^p_2, \ldots, \alpha_n^p)}{p}
        \]
        is a symmetric polynomial in the $\alpha$, with integral coefficients, and hence by the symmetric function theorem the 
        quantity must be a rational integer.
    \end{proof}

    \begin{lemma}
        Put $f(x) = x^n - 1$. Then $\overline{f}$ is a squarefree if and only if $p \nmid n$.
    \end{lemma}
    \begin{proof}
        By previous lemma we can see that if $p \nmid n$. Then $\gcd(\overline{f}, \overline{f}') = 1$, and hence 
        that $\overline{f}$ is squarefree. On the other hand, if $p|n$, say $n = mp$ for some integer $m$, then 
        \[
            \overline{f} = \overline{(x^m - 1)^p}
        \]
        and hence $\overline{f}$ is not squarefree.

        Let $\Phi_n(x)$ denote the $n$-th cyclotomic polynomial. Since $\Phi_n | f$, it follows from the above 
        that if $p \nmid n$, then $\overline{\Phi_n}$ is also squarefree.
    \end{proof}

    \begin{theorem}
        (Kronecker, 1854) The polynomial $\Phi_n(x)$ is irreducible over $\mathbb{Q}$.
    \end{theorem}
    \begin{proof}
        Suppose that $A$ and $B$ are monic polynomials with rational coefficients such that
        $\Phi_n = AB$, and suppose also that $\deg A > 0$. We know that A and B
        have integral coefficients. Let $Z$ denote the roots of $A$. Let $C$ be the monic polynomial
        whose roots are the numbers $\zeta^p$ for $\zeta \in Z$. Here $p$ is an arbitrary prime not dividing $n$.
        Our first step is to show that $A = C$. 
        
        Since the map $\zeta \mapsto \zeta^p$ merely permutes the roots of
        $\Phi_n$, we know that $C|\Phi_n$. Let $G = \gcd(B,C)$. Then $\overline{G}|B$ and $\overline{G}|C$. But $\overline{A} = \overline{C}$ by previous lemma,
        and hence $\overline{G}^2 | \olsi{A} \olsi{B}$. But $\Phi_n$ is squarefree, by previous lemma. Hence $G = \olsi{1}$, so $G = 1$,
        and consequently $C |A$. But $C$ and $A$ have the same degree, so in fact $A = C$.
        
        Now let $\zeta$ be a root of $A$, and $\zeta'$ a root of $\Phi_n$. Then there exists a positive integer $a$,
        $\gcd(a, n) = 1$, such that $\zeta' = \zeta^a$. We factor $a$, $a = p_1p_2 \ldots p_k$. Since $\zeta$ is a root of $A$, it follows
        from the argument above that $\zeta^{p_1}$ is also a root of A. Then by a second application of
        the above argument, we see that $\zeta^{p_1p_2}$ is also a root of $A$. Continuing in this manner, we
        deduce that $\zeta'$ is a root of $A$. Since this is valid for every root $\zeta'$ of $\Phi_n$, we conclude that
        $A = \Phi_n$. Hence $\Phi_n$ is irreducible.
    \end{proof}
\end{proof}



\begin{theorem}
    Let $F$ be a field and let $p(x) \in F[x]$. Then $\langle p(x) \rangle$ is a maximal ideal in $F[x]$
    if and only if $p(x)$ is irreducible over $F$.
\end{theorem}
\begin{proof}
    Suppose $\langle p(x) \rangle$ is a maximal ideal in $F[x]$. We know that $p(x) \neq 0$ and 
    $p(x)$ is not a unit since neither $\{ 0 \}$ nor $\langle 1_F \rangle = F[x]$ is a 
    maximal ideal in $F[x]$. Let 
    \[
        p(x) = g(x) h(x)
    \]
    be a factorization. Then $\langle p(x) \rangle \subseteq \langle g(x) \rangle 
    \subseteq \langle F[x] \rangle$ and since $\langle p(x) \rangle$ is maximal we either have 
    $\langle g(x) \rangle = \langle p(x) \rangle$ or $\langle g(x) \rangle = F[x]$. In the 
    first case we get 
\end{proof}

\begin{theorem}[Fundamental Theorem of Algebra]
    Every nonconstant polynomial in $\mathbb{C}[x]$ has a root in $\mathbb{C}$.
\end{theorem}
\begin{remark}
    The field $\mathbb{C}$ is algebraically closed.
\end{remark}

\begin{corollary}
    A polynomial is irreducible in $\mathbb{C}[x]$ if and only if it has a degree 1.
\end{corollary}
\begin{proof}
    All linear equation with degree 1 only have one root in $\mathbb{R}$.
\end{proof}

\begin{corollary}
    Every nonconstant polynomial $f(x)$ of degree $n$ can be written in the form 
    \[
        c(x-a_1)(x-a_2) \ldots (x-a_n)
    \]
    for some $c, a_1, a_2, \ldots, a_n \in \mathbb{C}$. This factorization is unique except for the order 
    of the factors.
\end{corollary}
\begin{proof}
    By the fundamental theorem of algebra, 
    \begin{align*}
        f(x) &= (r_1x + s_1)(r_2x + s_2) \ldots (r_nx + s_n)\\
        &= r_1\, r_2 \ldots r_n (x + s_1r_1^{-1})(x + s_2r^{-1}_2) \ldots (x + s_nr^{-1}_n).
    \end{align*}
    Since $f(x)$ has $n$ unique roots, factorization is also unique.
\end{proof}

\begin{lemma}
    If $f(x)$ is a polynomial in $\mathbb{R}[x]$ and $a + bi$ is a root of $f(x)$ in $\mathbb{C}$, 
    then $a - bi$ is also a root of $f(x)$.
\end{lemma}
\begin{proof}
    Let $z = a+bi$ and the conjugate $\olsi{z} = a - bi$. Define a map 
    $\varphi: \mathbb{C}[x] \to \mathbb{C}[x]$ by $\varphi f(x) = \olsi{f}$. 
    Bijective is trivial in $\varphi$, e.g.
        $\varphi(f + g) = \olsi{f} + \olsi{g}$
    and $\varphi(fg)= \olsi{f} \, \olsi{g}$ since 
    \[
        \overline{(a+bi)(c+di)} = \overline{(a+bi)}\> \overline{(c+di)}.
    \]
    If $f(x)$ has a root $z$ then $\olsi{f}(x)$ will has a root $\olsi{z}$. 
    If coefficients of $f(x)$ are all real numbers, then $f(x) = \olsi{f}(x)$. 
    Thus $f(x)$ has a root $\olsi{z}$.
\end{proof}

\begin{theorem}
    A polynomial $f(x)$ is irreducible in $\mathbb{R}[x]$ \text{if and only if} $f(x)$ is a 
    first-degree polynomial or $f(x) = ax^2 + bx + c$ with $b^2 - 4ac < 0$.
\end{theorem}
\begin{proof}
    In $\mathbb{C}[x]$,
    \[
        f(x) = c(x - a_1)(x-a_2)\ldots (x-a_n).
    \]
    If $a_i = c + di, a_j = c-di$ for some $1 \leq j \leq n$. The product of the conjugates are 
    \[
        (x - a_i)(x - a_j) = (x - c -di)(x - c + di) = x^2 -2cx + c^2 + d^2 \in \mathbb{R}[x].
    \]
    Thus we can pair them and so $f(x)$ can be split by irreducible polynomials whose degree is either 
    1 or 2.

    Now we knew every irreducible polynomial has a degree 1 or 2. When its degree is 2, then 
    \[
        f(x) = ax^2 + bx + c \quad \forall a,b,c \in \mathbb{R} \label{eq:r2.1} \tag{{\color{myred} $\clubsuit$}}
    \]
    We now continue to work on the "formula" to solve $x$. Completing the square on \eqref{eq:r2.1}
    \begin{align*}
        ax^2 + bx + c = 0 &\Rightarrow a \left[ x^2 + \frac{b}{a}x + \left(\frac{b}{2a} \right)^2 \right] - \left(\frac{b}{2a} \right)^2 = 0\\
        &\Rightarrow \left(x + \frac{b}{2a} \right)^2 = \frac{b^2 - 4ac}{4a^2}\\[0.35em]
        &\Rightarrow x + \frac{b}{2a} = \pm \frac{\sqrt{b^2 - 4ac}}{2a}\\[0.35em]
        &\Rightarrow x = \frac{-b \pm \sqrt{b^2 - 4ac}}{2a}, \quad a \neq 0\\
    \end{align*}
    Now we can take a look on determinant $\Delta = b^2 - 4ac$. If $\Delta < 0$, the two roots 
    will be in $\mathbb{C} \setminus \mathbb{R}$, else the two roots are in $\mathbb{R}$. (Either $\Delta > 0$ or 
    $\Delta = 0$). Hence the first-degree polynomial or quadratic polynomial is irreducible in $\mathbb{R}[x]$.
\end{proof}

\begin{corollary}
    Every polynomial $f(x)$ of odd degree in $\mathbb{R}[x]$ has a root in $\mathbb{C}$.
\end{corollary}
\begin{proof}
    Consequently, we can tell if a polynomial in $\mathbb{R}[x]$ or $\mathbb{C}[x]$ is irreducible without any elaborate 
    tests.
\end{proof}

\section{Integral Domains}

Let $R$ be a commutative ring. A \bred{zero divisor} is a nonzero element $a \in R$ such that 
\begin{equation}
    ab=0
\end{equation}
for some nonzero $b \in R$. The most familiar integral domain is $\mathbb{Z}$. It is a 
commutative ring with unity one. If $a, b \in \mathbb{Z}$ and $ab=0$, then either $a=0$ or 
$b=0$.

\begin{definition}
    A ring with unity $1$ having no zero divisors is an integral domain.
\end{definition}

\begin{lemma}
    Fields are integral domain
\end{lemma}
\begin{proof}
    Let $F$ be a field. We want to show that $F$ has no zero divisors. Suppose $ab =0$ and 
    $a \neq 0$. Then $a$ must has an inverse $a^{-1}$ such that $a^{-1}\, a b = a^{-1} \cdot 0 \Longrightarrow b = 0$.
    Therefore, $F$ has no zero divisors, and so $F$ is an integral domain.
\end{proof}

\begin{corollary}
    $\mathbb{Z}_p$ is a field.
\end{corollary}
\begin{proof}
    According to previous theorem, we need only prove that $\mathbb{Z}_p$ has no zero 
    divisors.

    Suppose that $a, b \in \mathbb{Z}_p$ and $ab = 0$. Then $ab = pk$ for some 
    integer $k$. But then we have either $p \, | \, a$ or $p p \, | \, b$ and in 
    $\mathbb{Z}_p$ is equivalent to either $a = 0$ or $b = 0$.
\end{proof}

\begin{lemma}
    If $R$ is an integral domain, then the characteristic of $R$ is either $0$ or a positive prime.
\end{lemma}
\begin{solution}
    Suppose not, suppose $R$ has characteristic $n=ab$ with $1 < a < b < n$. Then 
    \[
        (\underbrace{1_R + \cdots + 1_R}_{a \text{ times}})(\underbrace{1_R + \cdots + 1_R}_{b \text{ times}}) 
        = ab1_R = n1_R = 0.
    \]
    Since $R$ is an integral domain, it is either $a \cdot 1_R = 0$ or $b \cdot 1_R = 0$. This is contradicts 
    with our assumption that $n$ is not prime.
\end{solution}

\begin{definition}
    If $F$ is a field, then the only ideals are $\{0\}$ and $F$ itself.
\end{definition}
\begin{proof}
    Let $F$ be a field, and let $I \subset F$ be an ideal. Assume $I \neq \{0 \}$, and find $x \neq 0 \in I$. 
    Since  $F$ is a field, $x$ is invertible; Since $I$ is an ideal, $1 = x^{-1} \cdot x \in I$. Therefore $I = F$.
\end{proof}

\begin{example}
    The extended ring
    \[
        \mathbb{Q}[\sqrt{2}] = \{ a+b\sqrt{2} \> | \> a,b \in \mathbb{Q} \}
    \]
    is a field and that every nonzero element has a multiplicative inverse.
\end{example}
\begin{solution}
    This is clearly a ring. To show that every nonzero element has a multiplicative inverse. 
    Consider $a + b\sqrt{2} \neq 0 \in \mathbb{Q}[\sqrt{2}]$. The multiplicative inverse is 
    \[
        \frac{1}{a + b\sqrt{2}}
    \]
    Then multiplying top and bottom by conjugate, we have 
    \[
        \frac{a - b\sqrt{2}}{(a + b\sqrt{2})(a - b\sqrt{2})} = \frac{a - b\sqrt{2}}{a^2 - 2b^2}.
    \]
    Now we want to show $a^2 - 2b^2 \neq 0$.

    If $a = 0$ and $b \neq 0$ or if $a \neq 0$ and $b=0$, then $a^2 - 2b^2 \neq 0$. Since 
    $a^2 - 2b^2 \neq 0$, the only other possibility is $a, b \neq 0$.

    Thus, $a^2 = 2b^2$ with $a,b \neq 0$. We may assume that $a$ and $b$ are integers -- 
    in fact, now we can see $2$ divides $2b^2$, so $2 \,| \, a^2 \Longrightarrow \> 2 \, | \, a$. So 
    $a = 2c$ for some integer $c$. Plugging in gives $4c^2 = 2b^2 \Longrightarrow 2c^2 = b^2$.

    It follows that every nonzero element of $\mathbb{Q}[\sqrt{2}]$ is invertible, so 
    $\mathbb{Q}[\sqrt{2}]$ is a field.
\end{solution}

\begin{example}[Non-example]
    $\displaystyle M_2(\mathbb{Z}) = \left\{ \begin{bmatrix}
    a & b\\ c & d
    \end{bmatrix} \> \bigg \vert \> a,b,c,d \in \mathbb{Z} \right\}$ is not an integral domain.
\end{example}
\begin{solution}
    Choose
    \[
        A = \begin{bmatrix}
            1 & 0\\ 0 & 0
        \end{bmatrix}, \quad 
        B = \begin{bmatrix}
            0 & 0\\ 1 & 0
        \end{bmatrix}
    \]
    from $M_2(\mathbb{Z})$, and compute the matrix product
    \[
        AB = \begin{bmatrix}
            1 & 0\\ 0 & 0
        \end{bmatrix}
        \begin{bmatrix}
            0 & 0\\ 1 & 0
        \end{bmatrix} = \begin{bmatrix}
            0 & 0\\ 0 & 0
        \end{bmatrix} = \mathbf{0}.
    \]
    $A, B$ are zero divisors but none of them are zero. Thus $M_2(\mathbb{Z})$ is not an integral domain. 
\end{solution}

\begin{theorem}
    A finite integral domain is a field
\end{theorem}
\begin{proof}
    Let $D$ be a finite integral domain. Since $D$ is an integral domain, then $D$ is a commutative 
    ring with unity, and hence we need to show that $D$ is a field. In order to do this, we 
    want to show $\forall a \neq 0 \in D, \> \exists a^{-1} \in D$ such that 
    \[
        a \cdot a^{-1} = 1_D = a^{-1} \cdot a.
    \]
    Without loss of generality, we let 
    \[
        D = \{ a, a^2, a^3, \ldots, a^t \} 
    \]
    where $a \neq 0$ for some $t \in \mathbb{N}$. Consider two elements $a^i, a^j$ from $D$, we have 
    \begin{align*}
        a^i = a^j &\Rightarrow a^{i-j} = 1_D\\
        &\Rightarrow a\, a^{i-j-1} = 1_D\\
        &\Rightarrow a^{-1} = a^{i-j-1}\\
        &\Rightarrow a \cdot a^{-1} = a^{i-j} = 1_D.
    \end{align*}
    and the multiplication is commutative, therefore $D$ is a field.
\end{proof}
\begin{remark}
    \begin{align*}
        \textsf{if } \mathbb{Z}_p \textsf{ is a field} &\Longrightarrow \mathbb{Z}_p \textsf{ has no zero divisors.}\\
        &\Longrightarrow \mathbb{Z}_p \textsf{ is an integral domain.}\\
        &\Longrightarrow |\mathbb{Z}_p| = p \textsf{ is prime and is finite.}
    \end{align*}
\end{remark}

\begin{theorem}
    Let $R$ be a commutative ring with unity and let $A$ be an ideal of $R$. Then 
    $R/A$ is an integral domain if and only if $A$ is prime.
\end{theorem}
\begin{proof}
    We are going to prove in two directions:

    $(\Rightarrow)$ Suppose $R/A$ is an integral domain and $ab \in A$. Then 
    \[
        (a + A)(b + A) = ab + A = A
    \]
    is the zero element of the ring $R/A$. So it is either $a + A = A$ 
    or $b + A = A$; that is, either $a \in A$ or $b = A$. Thus $A$ is prime. 
    
    $(\Leftarrow)$ On the other hand, observe that $R/A$ is a commutative ring with 
    unity for any proper ideal $A$. Thus, we only need to show that when $A$ 
    is prime, $R/A$ has no zero divisors.

    Now suppose that $A$ is prime and 
    \[
        (a + A) (b+A) = 0 + A = A.
    \]
    Then $ab \in A$, and therefore, it is either $a \in A$ or
    $b \in A$. Thus, one of 
    $a + A$ or $b + A$ is the zero coset in $R/A$. Now we are done. 
\end{proof}

\section{Principal Ideal Domain}

\begin{definition}
    An integral domain $R$ is called a \bred{principal ideal domain} (or \bred{PID}) if every ideal in $R$ is principal.
\end{definition}

\begin{example}
    The integers $\mathbb{Z}$ and polynomial rings over fields are principal ideal domains.
\end{example}

\begin{theorem}
    If $F$ is a field then $F[x]$ is a PID.
\end{theorem}
\begin{proof}
    We know $F[x]$ is integral domain since $F$ is an integral domain. Let $I$ be an ideal of 
    $F[x]$.

    \textbf{Case 1}: If $I = \{ 0 \}$ then $I = \langle 0 \rangle$ and we are done.

    \textbf{Case 2}: If $I \neq \{ 0 \}$ let $g(x)$ be a nonzero polynomial of minimal degree in 
    $I$ (which exists by well-ordering). If $g(x)$ is constant then $g(x) = \alpha \in F$ and then 
    $I = F = \langle \alpha \rangle$ because for any $r \in F$ we have 
    \[
        r = r\alpha^{-1} \alpha \in \langle \alpha \rangle.
    \]
    Suppose then that $g(x)$ is not constant, we claim $I = \langle g(x) \rangle$. Since 
    $g(x) \in I$ we have $\langle g(x) \rangle \subseteq I$. We claim $I \subseteq \langle g(x) \rangle$. 
    Let $f(x) \in I$. By the \textit{division alogrithm}, we can write 
    \[
        f(x) = q(x)g(x) + r(x)
    \]
    with $0 \leq deg(r(x)) < deg(g(x))$. Since 
    \[
        r(x) = f(x) - q(x)\, g(x)
    \]
    we have $r(x) \in I$ and the fact that $g(x)$ is a nonzero polynomial of minimal degree implies 
    that $r(x) = 0$ and so $f(x) = q(x)\, g(x) \Longrightarrow  f(x) \in \langle g(x) \rangle$.
\end{proof}

\newpage
\section{Unique Factorization Domain}

\begin{definition}
    An integral domain $D$ is a \bred{unique factorization domain} (\bred{UFD} in short) 
    if 
    \begin{enumerate}
        \item Every nonzero element of $D$ that is not a unit can be written as a product of 
        irreducibles of $D$, and
        \item The factorization into irreducibles is unique up to associates and the order in which 
        the factors appear. 
    \end{enumerate}
\end{definition}

\begin{theorem}
    Every PID is a UFD.
\end{theorem}
\begin{proof}
    Let $R$ be a PID and suppose that a nonzero element $a$ of $R$ can be express in two different ways 
    as a product of irreducibles. Suppose
    \[
        a = p_1 p_2 \cdots p_r \quad \text{and } a = q_1 q_2 \cdots q_s 
    \]
    where each $p_i$ and $q_j$ is irreducible in $R$, and $s \geq r$. Then $p_1$ divides the 
    product $q_1, q_2, \cdots, q_s $ and so $p_1 | q_j$ for some $j$, as $p_1$ is prime. 
    After reordering the $q_j$ we can consider $p_1|q_1$. Then $q_1 = u_1\, p_1$ for some unit $u_1$ of $R$,
    since $q_1$ and $p_1$ are both irreducible. Thus
    \[
        p_1 p_2 \cdots p_r = u_1 p_1 q_2 \cdots q_s
    \]
    and cancelling $p_1$ on both side
    \[
        p_2 \cdots p_r = u_1 q_2 \cdots q_s.
    \]
    Continuing this process we reach
    \[
        1 = u_1 u_2 \ldots u_r\, q_{r+1} \ldots q_s.
    \]
    Since none of the $q_j$ is a unit, this means that $r=s$ and $p_1 p_2 \cdots p_r$ are associates of 
    $q_1 q_2 \cdots q_r$ in some order. Thus $R$ is a unique factorization domain.
\end{proof}

\begin{theorem}
    Every field is a UFD.
\end{theorem}
\begin{proof}
    Every field $F$ is a UFD because it is PID $\Rightarrow$ it is an integral domain and every nonzero 
    is a unit, and it contains no prime.
\end{proof}

\section{Euclidean Domains}

\begin{definition}
    An integral domain $D$ is called a \bred{Euclidean Domain} if there is a function $d$ 
    (called the measure) from the nonzero elements of $D$ to the positive integers such that 

    \begin{enumerate}
        \item $d(a) \leq d(ab)$ for all nonzero $a,b \in D$; and 
        \item If $a,b \in D, b \neq 0$, then there exists elements $q, r \in D$ such that 
        \[
            a = bq + r
        \]
        where $r = 0$ or $d(r) < d(b)$.
    \end{enumerate}
\end{definition}

\begin{example}
    The ring $\mathbb{Z}$ is a Euclidean domain with $d(a) = |a|$.
\end{example}

\begin{example}
    Let $F$ be a field. Then $F[x]$ is a Euclidean domain with $d(f(x)) = \deg\, f(x)$.
\end{example}

\begin{example}
    Gaussian integers $\mathbb{Z}[i]$ is a Euclidean domain with
    \[
        d(a+bi) = a^2 + b^2.
    \]
\end{example}

\begin{theorem}
    Every Euclidean domain is a PID.
\end{theorem}
\begin{proof}
    Let $E$ be a Euclidean domain. Consider an ideal $I$ of $E$. If $I = \{ 0 \}$, then 
    $I = \langle 0 \rangle$.

    Let $I \neq \{ 0 \}$. Then $N = \{ d(x) \> | \> x \in I, x \neq 0 \}$ is a nonempty 
    set of nonnegative integers; and so, by the well-ordering principle 
    it has the least element.

    Let $a \in I, a \neq 0$ such that $d(a)$ is the least element of $N$. i.e. $d(a) \leq d(x)$ for all 
    nonzero $x$ in $I$. We want to show $I = Ea$. Since $a \in I$, it follows that 
    \[
        Ea \subseteq I.
    \]
    Let $b \in I$. Since $E$ is a Euclidean domain, there exist $q, r \in E$ such that 
    \[
        b = aq+r, \quad \text{where } r = 0 \text{ or } d(r) < d(a).
    \]
    If $r \neq 0$, then $r = b - aq \in I$ shows that $d(r) \in N$; and since 
    $d(r) < d(a)$, this contradicts the minimality of $d(a)$ in $N$. Therefore, $r = 0$ and so $b = aq \in Ea$.
    Thus $I \subseteq Ea$ and hence $I = Ea$.
\end{proof}

\begin{remark}
    If $D$ is Euclidean domain, then it is both PID and UFD.
\end{remark}

\begin{theorem}
    If $D$ is UFD, then $D[x]$ is UFD.
\end{theorem}
\begin{proof}
    Let $D$ be an Euclidean domain and $\varnothing \neq I \triangleleft D$.

    Among all the nonzero elements of $I$, let $a \in I$ be such that $d(a)$ is a minimum. 
    Then
    \[
        I = \langle a \rangle.
    \]
    For if $b \in I$, there exist some $q,r$ such that 
    \[
        b = aq+r,
    \]
    where $r=0$ or $d(r) < d(a)$. But $r = b-aq \in I$, so $d(r)$ cannot be less than $d(a)$. 
    Thus, $r = 0$ and $b = aq \in \langle a \rangle$
\end{proof}

\section*{Tutorials}

\begin{mdframed}
    \vspace{-0.25cm}
    \hspace{-0.25cm}
    \begin{Exercise}
        For each of the following, decide whether the indicated operations on the set will form a ring. If a ring is not formed, 
        state the reason why this is the case. If a ring is formed state whether the ring is commutative, whether it has unity, 
        and whether it is a field. 
        \begin{enumerate}
            \item $n\mathbb{Z}$, under the usual addition and multiplication.
            \item $n\mathbb{R}^+$, under the usual addition and multiplication.
            \item $n\mathbb{Z} \times \mathbb{Z}$ with addition and multiplication by components.
            \item $n\mathbb{Z} \times 2\mathbb{Z}$ with addition and multiplication by components.
            \item $\{ a + b\sqrt{5} \> | \> a,b \in \mathbb{Q} \}$ with the usual addition and multiplication.
            \item $\{ ri \> | \> r \in \mathbb{R} \}$ with the usual addition and multiplication where $i^2 = -1$.
        \end{enumerate}
    \end{Exercise}

    \vspace{0.752cm}
    \begin{Exercise}
        Let $\alpha = \sqrt[3]{5}$ and $\mathbb{Z}[\alpha] = \{ a + b\alpha + c\alpha^2 \> | \> a,b,c \in \mathbb{Z} \}$. Prove whether 
        $\mathbb{Z}[\alpha]$ is a subring of $\mathbb{R}$.
    \end{Exercise}

    \vspace{0.752cm}
    \begin{Exercise}
        Let $X$ be some arbitrary set, and $P(X)$ be the set of all subsets of $X$. Define operators on $P(X)$ as follows, 
        where $a,b$ in $P(X)$:
        \[
            a + b = (a \cup b) \setminus (a \cap b)
        \]
        and 
        \[
            ab = a \cap b.
        \]
        Show that $P(X)$ is a commutative ring.
    \end{Exercise}

    \vspace{0.752cm}
    \begin{Exercise}
        Let $\mathbb{A}$ be the set $\mathbb{A} = \{ a + bi \> | \> a, b \in \mathbb{Q} \}$
        where $i^2 = -1$. Here,
        \[
            (a + bi) + (c + di) = (a + c) + (b+d)i
        \] 
        and 
        \[
            (a+bi)(c+di) = (ac-bd) + (ad - bc)i.
        \]
        Show that $\mathbb{A}$ is a field.
    \end{Exercise}

    \vspace{0.752cm}
    \begin{Exercise}
        Show that the rings $2\mathbb{Z}$ and $3\mathbb{Z}$ are not isomorphic.
    \end{Exercise}

    \vspace{0.752cm}
    \begin{Exercise}
        Show that a ring $R$ has no nonzero nilpotent element if and only if $0$ is the only solution of $x^2 = 0$ in $R$.
    \end{Exercise}

    \vspace{0.752cm}
    \begin{Exercise}
        Show that if $R$ is a ring with unity and $N$ is an ideal of $R$ such that $N \neq R$, then $R/N$ is a ring with unity.
    \end{Exercise}

    \vspace{0.752cm}
    \begin{Exercise}
        If $F$ is a field, show that $(F \setminus \{0\}, \cdot)$ is a group.
    \end{Exercise}

    \vspace{0.752cm}
    \begin{Exercise}
        Show that in a field $F$, the only ideals are $F$ and $\{ 0 \}$.
    \end{Exercise}

    \vspace{0.752cm}
    \begin{Exercise}
        Show that each homomorphism from a field to a ring is either one to one or maps everything onto 0. 
    \end{Exercise}

    \vspace{0.752cm}
    \begin{Exercise}
        Find the characteristic of the following rings:
        \begin{enumerate}
            \item $2 \mathbb{Z}$.
            \item $\mathbb{Z}_3 \times 3 \mathbb{Z}$.
            \item $\mathbb{Z}_5 \times \mathbb{Z}_5$.
        \end{enumerate} 
    \end{Exercise}

    \vspace{0.752cm}
    \begin{Exercise}
        Show that the matrix $\begin{bmatrix}
            1 & 2 \\ 2 & 4
        \end{bmatrix}$ is a zero divisor in $M_2(\mathbb{Z})$.
    \end{Exercise}

    \vspace{0.752cm}
    \begin{Exercise}
        An element $\mathfrak{a}$ of a ring $R$ is idempotent if $\mathfrak{a}^2 = \mathfrak{a}$. Show that a division ring 
        contains exactly two idempotent elements.
    \end{Exercise}

    \vspace{0.752cm}
    \begin{Exercise}
        If $A$ and $B$ are ideals of a ring $R$, then the sum $A + B$ of $A$ and $B$ is defined by 
        \[
            A + B = \{ a + b \> | \> a \in A, b \in B \}.
        \]
        \begin{enumerate}
            \item Show that $A + B$ is an ideal of $R$.
            \item Show that $A \subseteq A + B$.
        \end{enumerate}
    \end{Exercise}

    \vspace{0.752cm}
    \begin{Exercise}
        If $A$ and $B$ are ideals of a ring $R$, then the product $AB$ of $A$ and $B$ is defined by 
        \[
            AB = \left\{ \sum^n_{i=1} a_ib_i \> \bigg | \> a_i \in A, b_i \in B, n \in \mathbb{Z}^+ \right\}.
        \]
        \begin{enumerate}
            \item Show that $AB$ is an ideal of $R$.
            \item Show that $AB \subseteq (A \cap B)$.
        \end{enumerate}
    \end{Exercise}

    \vspace{0.752cm}
    \begin{Exercise}
        Find $q(x)$ and remainder $r(x)$ as described by the division algorithm so that 
        \[
            f(x) = g(x)q(x) + r(x)
        \]
        with $r(x) = 0$ or of degree less than the degree of $g(x)$.
        \begin{enumerate}
            \item $f(x) = x^6 + 3x^5 + 4x^2 - 3x + 2$ and $g(x) = x^2 + 2x - 3$ in $\mathbb{Z}_7[x]$.
            \item $f(x) = x^5 - 2x^4 + 3x - 5$ and $g(x) = 2x + 1$ in $\mathbb{Z}_{11}[x]$.
        \end{enumerate}
    \end{Exercise}
\end{mdframed}