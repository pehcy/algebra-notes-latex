\chapter{Fields}

\section{Extenstion Fields}

\begin{definition}[Extension field]
    An extension field $K$ of a field $F$ is said to be an algebraic extension of $F$ if every element 
    of $K$ is algebraic over $F$.
\end{definition}

\begin{example}
    $\mathbb{C}$ is an algebraic extension of $\mathbb{R}$. $\forall a+bi \in \mathbb{C}$, where 
    $a,b \in \mathbb{R}$ and $i = \sqrt{-1}$. We have 
    \[
        (x+a+bi)(x+a-bi) = x^2 +2ax + a^2 + b^2.
    \]
    Thus $a+bi$ is a root of $x^2 +2ax + a^2 + b^2 = 0$.
\end{example}

\begin{theorem}
    If $K$ is a finite-dimensional extension field of $F$, then $K$ is an algebraic extension of $F$.
\end{theorem}
\begin{proof}
    Let $\{ V_1, V_2, \ldots, V_n \}$ be the basis of $K$ over $F$. For all $u \in K$, 
    $\{ 1, u, u^2, \ldots, u^n \}$ is linearly dependent. That is, 
    \[
        \exists u^k \in K \> s.t. \> u^k = \text{Span} \{1, u, u^2, \ldots, u^n\} = c_0 + c_1 u + c_2 u^2 +
        \cdots + c_{k-1} u^{k-1} (k \geq 1).
    \]
    Thus $u$ is a root of $f(x) = x^k - c_{k-1} u^{k-1} - \cdots - c_0$, this implies $K$ is an algebraic extension.
\end{proof}

\hrule
\vspace{10pt}

In fact, a simple extension is an algebraic extension if $u$ is algebraic. If extension field $K$ contains a 
transcental element $u$, then $K$ must be infinite dimensional over $F$.
\[
    \textsf{Non algebraic} \Longrightarrow \textsf{Infinite dimension}
\]
Note that $F(u)$ denote the intersection of all subfields of $K$ that contains both $F$ and $u$. It said to be a simple 
extension of $F$. If $u_1, u_2, \ldots, u_n$ are elements of an extension field $K$ of $F$. Let 
$F(u_1, \ldots, u_n)$ denote the intersection of all the subfields of $K$ that contain $F$ and every $u$ (known as 
generalized simple extension); $F(u, u_1, \ldots, u_n)$ is said to be a finitely generated extension of $F$.

\begin{corollary}
    If $L$ is algebraic extension of $K$ and $K$ is an algebraic extension of $L$. Then 
    $L$ is an algebraic extension field of $F$.
\end{corollary}
\begin{proof}
    $\forall \omega \in L, \> \exists f(x) \in K[x] \quad s.t. \> f(\omega) = a_0 + a_1 \omega 
    + \cdots + a_n \omega^n$. 

    Note that $F(a_0, a_1, \ldots, a_n)$ is finitely generated extension of $F$ and all $a_i$'s
    \\are algebraic. Thus it is finite dimensional algebraic extension of $F$. Since 
    $\omega$ is algebraic over $F(a_0, a_1, \ldots, a_n)$. So $F(a_0, a_1, \ldots, a_n)$ is finite 
    dimensional extension of $F \Longrightarrow \omega$ is algebraic over $F$.
    Thus $L$ is an algebraic extension of $F$.
\end{proof}

\begin{remark}
    Algebraic subfield $E$ of $\mathbb{C}$ over $\mathbb{Q}$ is called the \textbf{field of algebraic numbers}. 
    Where $E$ is an finite-dimensional algebraic extension over $K$.
    \[
        \begin{tikzcd}[row sep=tiny]
            \mathbb{C} &\\
             & E  \arrow[dash]{ul}
            \\
            \mathbb{Q} \arrow[uu, "\pi"] \arrow[ur, "\mu"]&\\
        \end{tikzcd}
    \]

    \begin{itemize}
        \item $\mu$ denote algebraic extension over $\mathbb{Q}$, e.g.: $\sqrt{2}, \sqrt{3}, i, \ldots$.
        \item $\pi$ denote non-algebraic extension.
    \end{itemize}
\end{remark}

\begin{corollary}
    Let $K$ be an extension field of $F$ and let $E$ be the set of all elements of 
\end{corollary}


\begin{example}
    \[
        \mathbb{Q}(i, -i) = \mathbb{Q}(i)
    \]
\end{example}

\begin{example}
    $$\mathbb{Q}(\sqrt{3}, i) = \mathbb{Q}(\sqrt{3})(i)$$
\end{example}
\begin{solution}
    \begin{align*}
        [\mathbb{Q}(\sqrt{3}, i)] &= [\mathbb{Q}(\sqrt{3})(i): \mathbb{Q}]\\
        &= [\mathbb{Q}(\sqrt{3})(i): \mathbb{Q}(\sqrt{3})]\, [\mathbb{Q}(\sqrt{3}): \mathbb{Q}]\\
        &= 2 \cdot 2\\
        &= 4
    \end{align*}
\end{solution}

\begin{example}
    Every finite-dimensional extension is also finitely generated. If $\{ u_1, u_2, \ldots, u_n \}$ 
    is a basis of $K$ over $F$. This implies $F(u_1, u_2, \ldots, u_n) \subseteq K$ and 
    $K \subseteq F(u_1, u_2, \ldots, u_n)$. 
    Thus,
    \[
        K = F(u_1, u_2, \ldots, u_n) = \text{Span}\{ u_1, u_2, \ldots, u_n \}.
    \]
\end{example}

\begin{example}[Non-example]
     $$\mathbb{Q}(\sqrt{3}, \sqrt{5}) \neq \mathbb{Q}(\sqrt{3})$$
\end{example}
\begin{solution}
    For the sake of contradiction, consider $\mathbb{Q}(\sqrt{3}, \sqrt{5}) = \mathbb{Q}(\sqrt{3})$, then 
    \[
        \sqrt{5} = a+ b\sqrt{3}, \quad \forall a,b \in \mathbb{Q}
    \]
    Altering this equation by moving $a$ to left-hand side, then squaring both sides. We obtain
    \begin{align*}
        (\sqrt{5} - a)^2 = (b\sqrt{3})^2 \quad &\Rightarrow 5 - 2\sqrt{5}a + a^2 = 3b^2\\
        &\Rightarrow \frac{5 + a^2 - 3b^2}{2a} = \sqrt{5} \quad (a\neq 0)
    \end{align*}
    However, when $a=0$, we have $5 = 3b^2$. Which is a contradiction.
\end{solution}

\section{Fundamental Theorem of Galois theory}

\begin{definition}[Galois correspondence]
    Let $K$ be a finite-dimensionalextension field of $F$, and let 
    $S$ be the set of all intermediate fields. Again we let 
    $T$ be the set of all subgroups of the Galois group $\gal_F K$.

    Define a map $\phi : T \to S$ by this rule. For each intermediate field $E$,
    \begin{equation}
        \phi(E) = \gal_E K.
    \end{equation} 
    This function $\phi$ is called the Galois correspondence.

    \begin{center}
        $\begin{tikzcd}
            \gal_K K \arrow[r] 
            & K \arrow[d, "S"] \\
            \gal_F K \arrow[u,"T"] \arrow[r]
            & F
            \end{tikzcd}
        $
    \end{center}
\end{definition}

\begin{example}
    \[
        \mathbb{Q} \rightarrow \gal_\mathbb{Q} \mathbb{Q}(\sqrt{3}, \sqrt{5}) = \{i, \tau, \alpha, \beta \}
    \]
\end{example}

\begin{lemma}
    Let $K$ be a finite-dimensional extension field of $F$. If $H$ is a subgroup of the Galois group 
    $\gal_F K$ and $E$ is the \bgold{fixed field} of $H$, then $K$ is 
    \textit{simple, normal, separable extension} of $E$.

    \begin{center}
        $\begin{tikzcd}
            \gal_F K  
            & K \arrow[dash]{d} \\
            H \arrow[dash]{u} \arrow[r]
            & E_H
            \end{tikzcd}
        $
    \end{center}
\end{lemma}
\begin{proof}
    Since $K$ is finite-dimensional extension field, so $K$ is algebraic over $F$. Let $\mathfrak{U} \in K$ and 
    $p(x) \in E[x]$ be minimal polynomial of $\mathfrak{U}$, and $\forall \sigma \in H$, $\sigma(\mathfrak{U})$ is 
    some root of $p(x)$.

    Therefore, $\mathfrak{U}$ has a finite number of distinct images under automorphisms in $H$, said 
    \[
        \mathfrak{U} = u_1, u_2, \ldots, u_t \in K, \quad \text{where } t \leq \deg\, p(x)
    \]
    If $\sigma \in H$ and $u_i = \tau(\mathfrak{U})$ with $\tau \in H$, then $\sigma(u_i) = \sigma \circ \tau (\mathfrak{U})$.

    Since $\sigma$ is injective, so 
    \[
        \mathfrak{U} \xhookrightarrow{H} \{ u_1, u_2, \ldots, u_t \}
    \]
    which $\{ u_1, u_2, \ldots u_t\}$ is image of $\mathfrak{U}$. And $u_i = \tau(\mathfrak{U})$ for some $\tau \in H$.
    is injective 
    \[
        \{ u_1, u_2, \ldots, u_t \} \xhookrightarrow[\text{permutation } \sigma]{} \{ u_1, u_2, \ldots, u_t \}
    \]
    Every automorphism in $H$ permutes $u_1, u_2, \ldots, u_t$. Let 
    \[
        f(x) = (x-u_1)(x-u_2)\ldots(x-u_t)
    \]
    Since all $u_i$'s are distinct, $f(x)$ is separable.

    Now we claim that $f(x) \in E[x]$. Note that $\sigma f(x) = f(x)$ for all $\sigma \in H$. All coefficients of 
    $f(x)$ is fixed by $\sigma \in H$. Thus $f(x) \in E[x]$. Since $u = u_1$ is a root of $f(x) \in E[x]$, $u$ is separable over 
    $E. \Longrightarrow \> K$ is separable extension of $E$.

    We state that $K = E(V)$ for some $V \in K$. 
\end{proof}