\chapter{Groups}

\begin{definition}
    A group $(G, *)$ is a set $G$, together with a binary operator $*$ such that 
\end{definition}

\begin{center}
    {
    \renewcommand{\arraystretch}{2}
    \begin{tabular}{| m{20em} | m{20em} |}
        \hline
        Additive Group & Multiplicative Group\\
        \hline
        Let $G$ be a set, and ${\color{red} +}$ be an operation, 
        then $(G, {\color{red} +})$ is an additive group provided & 
        Let $G$ be a set, and  be an operation, 
        then $(G, {\color{ForestGreen} \circ})$ is an multiplicative group provided\\[1em]
        \hline
        1. $\forall a, b \in G$, $a \, {\color{red} +}\, b \in G$ & 
        6. $\forall a, b \in G$, $a \,{\color{ForestGreen} \circ}\, b \in G$\\[1em]
        \hline
        2. $\forall a, b, c \in G$, $a \, {\color{red} +}\, (b \, {\color{red} +}\, c ) = (a \, {\color{red} +}\, b) \, {\color{red} +}\, c $ & 
        7. $\forall a, b, c \in G$, $a \, {\color{ForestGreen} \circ}\, (b \, {\color{ForestGreen} \circ}\, c ) = (a \, {\color{ForestGreen} \circ}\, b) \, {\color{ForestGreen} \circ}\, c $\\[1em]
        \hline
        3. $\forall a \in G, \exists\, 0 \in G $ (identity) s.t. 
            \[a \, {\color{red} +} \, 0 = a = 0 \, {\color{red} +}\, a\] & 
        8. $\forall a \in G, \exists 1 \in G $ (unity) s.t. \[a \, {\color{ForestGreen} \circ}\, 1 = a = 1 \, {\color{ForestGreen} \circ} \, a\]\\[1em] 
        \hline
        
        4. $\forall a \in G, \exists\, -a \in G $ (additive inverse) s.t. 
            \[a \, {\color{red} +}\, (-a) = 0 = (-a)\, {\color{red} +}\, a\] & 
        9. $\forall a \in G, \exists a^{-1} \in G $ (unity) s.t. \[a \, {\color{ForestGreen} \circ}\, a^{-1} = 1 = a^{-1} \, {\color{ForestGreen} \circ}\, a\]\\[1em]
        \hline
        5. (Commutative) $\forall a, b \in G$, $a \, {\color{red} +}\, b = b \, {\color{red} +}\, a$ &
        10. (Commutative) $\forall a, b \in G$, $a \, {\color{ForestGreen} \circ}\, b = b \, {\color{ForestGreen} \circ}\, a$\\[1em]
        \hline
    \end{tabular}
    }
\end{center}

Joining additive and multiplicative groups together, we form a ring with \textbf{distributive laws}

\[11. \quad \forall a, b, c \in G, (a \, {\color{red} +}\, b) \, {\color{ForestGreen} \circ}\, c = (a \, {\color{ForestGreen} \circ}\, c) \, {\color{red} +}\, (b \, {\color{ForestGreen} \circ}\, c)  \]
\[12. \quad \forall a, b, c \in G, c \, {\color{ForestGreen} \circ}\, (a \, {\color{red} +}\, b) = (c \, {\color{ForestGreen} \circ}\, a) \, {\color{red} +}\, (c \, {\color{ForestGreen} \circ}\, b)  \]

\begin{itemize}
    \item Abelian group: (1-5) or (6-10)
    \item Associative Ring: 1-6, with 11 and 12
    \item Semigroup: 1, 2 only
    \item Monoid: 1, 3 only
    \item Commutative ring: 1-5, 6, 10, 11, and 12
    \item Ring: 1-5, with 11 and 12
    \item Ring with unity: 1-6, with 8, 11, and 12
    \item Field: 1-12
\end{itemize}

\begin{lemma}[Uniqueness of group identity]
    In a group $G$, there is one and only one identity element $e$.
\end{lemma}
\begin{proof}
    \textit{For the sake of contradiction}. Suppose not, Suppose that $e$ and $e'$ are both identity elements of group $G$.
    Since $e$ is an identity element of $G$, then $e \in G$ and 
    \begin{equation*}
        ea= a = ae \quad \forall a \in G.  \label{eq:g1.1} \tag{{\color{red} $\heartsuit$}}
    \end{equation*}
    Since $e'$ is also an identity element of $G$. we said that $e' \in G$ and 
    \begin{equation*}
        e'a= a = ae' \quad \forall a \in G.  \label{eq:g1.2} \tag{{\color{cyan} $\clubsuit$}}
    \end{equation*}

    From \eqref{eq:g1.1}, if we take $a = e'$, then $e \cdot e' = e'$. 

    From \eqref{eq:g1.2}, if we take $a = e$, then $e = e \cdot e'$.
    
    Combining the results we have $e = e \cdot e' = e'$, and so $e = e'$. There is only one identity element 
    in $G$. We proved the uniqueness of identity.
\end{proof}

\begin{lemma}[Cancellation rule]
    In a group $G$, $ba = ca$ implies $b = c$; and $ab = ac$ implies $b=c$.
\end{lemma}
\begin{proof}
    Consider $G$ is a group, then 
    \[
        \forall a \in G, \exists a' \in G \quad s.t. \quad aa' = e = a'a.
    \]
    To show the right cancellation works, we further consider $ba=ca$. Multiplying $a'$ on both sides of the previous equation on right, 
    we obtained 
    \[
     (ba)a' = (ca)a'
    \]
    Then, $b(aa') = c(aa')$ and so $be = ce \Rightarrow \fbox{$b = c$}$. The proof is now complete.
\end{proof}

\begin{theorem}[Socks-shoes property]
    \begin{equation}
        ({\color{red} a}\, \circ \, {\color{BurntOrange} b })^{-1} = {\color{BurntOrange} b }^{-1} \, \circ \, {\color{red} a}^{-1}
    \end{equation}
\end{theorem}
\begin{proof}
    Since we know that $G$ is a group, then $ab \in G$ for all $a, b \in G$ since $G$ is closure.
    Next, we consider the following equation 

    \begin{align*}
        (ab)(b^{-1}\, a^{-1}) &= a(bb^{-1})a^{-1} & G \text{ is asscoiative}\\
        &= aea^{-1}\\
        &= aa^{-1}\\
        &= \fbox{$e$} & \text{cancellation rule returns identity}
    \end{align*}
    this equation states that
    \[
        \fbox{$(ab)(b^{-1}\, a^{-1}) = a(bb^{-1})a^{-1}$} = e
    \]
    now we cancel off $ab$ from both sides of the equations, we now arrive at 
    \[
        (ab)^{-1} = b^{-1} a^{-1}
    \]
    and we have done the proof.
\end{proof}

\begin{remark}
    In abstract algebra, the position of inputs in binary operator is very important! The commutative property no necessary hold.
    ${\color{red} a}\, \circ \, {\color{BurntOrange} b } \neq {\color{red} b}\, \circ \, {\color{BurntOrange} a }$.
    E.g. matrix multiplication $AB \neq BA$.
\end{remark}

\begin{example}
    Consider $(a,b)$ to be a fixed point on the 2-dimensional cartesian plane $\mathbb{R}^2$, we define a translation map
    $T_{a,b} : \mathbb{R}^2 \to \mathbb{R}^2$ such that 
    \[
        T_{a,b}(x,y) = (x+a, y+b)
    \]
    we again define $G = \{T_{a,b} \> | \> a, b \in \mathbb{R}\}$. Show that $(G, \circ)$ is a group under function composition.
\end{example}
\begin{solution}
    \begin{enumerate}
        \item (Closure) We want to show:
            \[
                \forall T_{a,b}, T_{c,d} \in G, \quad T_{a,b} \circ T_{c,d} \in G
            \]
        We compute the composition 
        \begin{align*}
            (T_{a,b} \circ T_{c,d})(x,y) &= T_{a,b}(T_{c,d}(x,y))\\
            &= T_{a,b}(x+c, y+d)\\
            &= (x+a+c,y+b+d)\\
            &= (x+(a+c), y+(b+d)) & \text{asscoiativity of ordinary addition}\\
            &= T_{a+c,\, b+d}(x,y)
        \end{align*}
        which closed under $G$.

        \item ()
    \end{enumerate}
\end{solution}

\begin{theorem}
    The following statements are equivalent.

    \begin{enumerate}
        \item Every subgroup of a cyclic group (multiplicative group) is cyclic.
        \item If $|\langle a \rangle| = n$, then the order of any subgroup of $\langle a \rangle$ is a divisor of $n$.
        \item For each positive divisor $k | n$, $\langle a \rangle$ has exactly one subgroup of order $k$. 
        $\langle a^{n/k} \rangle$ if multiplicative group, $\langle \frac{n}{k}a \rangle$ if additive group.
    \end{enumerate}
\end{theorem}
\begin{proof}
    Let $G$ be a cyclic group and $H$ be a subgroup of $G$. We need to show that $H$ is also cyclic.
    Example: $H = \langle a^m \rangle$ s.t. $m$ is the least positive integer.

    By randomly pick integer $b \in H$, $b = a^k, k \in \mathbb{Z}^+$. By division algorithm, $k = qm + r$, where $0 \leq r < m$.

    \begin{align*}
        b = a^k = a^{qm+r} = (a^m)^{q}a^r &\Rightarrow a^r = (a^m)^{-q} b \in H\\
        &\Rightarrow a^r \in H, \quad 0 \leq r < m\\
        &\Rightarrow r = 0
    \end{align*}
\end{proof}

\section{Subgroups}

\subsection{Subgroup tests}

\begin{theorem}[One step subgroup test]
    Suppose $G$ is a multiplicative group and $H \subseteq G$. If 
    \begin{enumerate}
        \item $H \neq \varnothing$,
        \item $\forall a, b \in H, ab^{-1} \in H$
    \end{enumerate}
    then $H$ is a subgroup of $G$. 
\end{theorem}

\begin{example}
    Let 
\end{example}

\section{Sylow's theorem}

\begin{theorem}
    $C_5 \times C_2$ and $C_{10}$ are two isomorphism classes.
\end{theorem}
\begin{proof}
    From the \textit{Third Sylow's theorem}, the number of Sylow 5-groups divides 2 and is $1 (mod 5)$, so there is only one Sylow 5-group.
    And there is a normal subgroup $K \trianglelefteq G$ such that $|K| = 5$.
\end{proof}

\section{Automorphism}

\begin{example}
    Compute $\text{Aut}(\mathbb{Z}_{10})$.
\end{example}
\begin{solution}
    For any $\alpha \in \text{Aut}(\mathbb{Z}_{10})$ and for any $k \in \mathbb{Z}_{10}$. We define $k \mapsto k \alpha(1)$ such that 
    \[
        1 \mapsto \alpha_1 :\quad \mathbb{Z}_{10} \to \mathbb{Z}_{10}, \quad \alpha_1(x) = x
    \]
    \[
        3 \mapsto \alpha_3 :\quad \mathbb{Z}_{10} \to \mathbb{Z}_{10}, \quad \alpha_3(x) = 3x
    \]
    \[
        7 \mapsto \alpha_7 :\quad \mathbb{Z}_{10} \to \mathbb{Z}_{10}, \quad \alpha_7(x) = 7x
    \]
    \[
        9 \mapsto \alpha_9 :\quad \mathbb{Z}_{10} \to \mathbb{Z}_{10}, \quad \alpha_9(x) = 9x
    \]

    In fact, $\text{Aut}(\mathbb{Z}_{10})$ is isomorphic to $U(10) = \{ 1, 3, 7, 9\}$.
\end{solution}

\section{Cosets}

\begin{example}
    Consider $G = \mathbb{Z}_9 = \{ 0, 1, 2, \ldots, 8 \} (\text{mod} \, 9)$. We take a cyclic subgroup 
    \[H = \langle 3 \rangle = \{0, 3, 6\}\] 
    which came from $(G, \oplus)$. All \textbf{left cosets} of $G$ with respect to $H$ are $\{ H, 1 \oplus H, 2 \oplus H \}$ where
    \begin{align*}
        &0 \oplus H = \{0 + 0, 0+ 3, 0 + 6\} (\text{mod} \, 9) = \{0, 3, 6\} = H\\
        &1H = 1 \oplus H = \{1 + 0, 1 + 3, 1 + 6\} (\text{mod} \, 9) = \{1, 4, 7\}\\
        &2H = 2 \oplus H = \{2 + 0, 2 + 3, 2 + 6\} (\text{mod} \, 9) = \{2, 5, 8\}\\
        &3H = 3 \oplus H = \{3 + 0, 3 + 3, 3 + 6\} (\text{mod} \, 9) = \{3, 6, 0\} = H
    \end{align*}

    As for the right cosets of $G$ with respect to $H$ are $\{ H, H \oplus 1, H \oplus 2 \}$. Pay attention that now the element of coset are being 
    added to right-hand side instead of from left side.

    \begin{align*}
        &0 \oplus H = \{0 + 0, 0+ 3, 0 + 6\} (\text{mod} \, 9) = \{0, 3, 6\} = H\\
        &H1 = H \oplus 1 = \{0 + 1, 3 + 1, 6 + 1\} (\text{mod} \, 9) = \{1, 4, 7\}\\
        &H2 = H \oplus 2 = \{0 + 2, 3 + 2, 6 + 2\} (\text{mod} \, 9) = \{2, 5, 8\}\\
        &H3 = H \oplus 3 = \{0 + 3, 3 + 3, 6 + 3\} (\text{mod} \, 9) = \{3, 6, 0\} = H
    \end{align*}
\end{example}

\section{Normal subgroups and Quotient groups}

\subsection{Normal subgroup}

\begin{definition}[Normal subgroups]
    A subgroup $H$ of $(G, \cdot)$ is called a normal subgroup if for all $g \in G$ we have 
    \begin{equation}
        gH = Hg.
    \end{equation}
    We shall denote that $H$ is a subgroup of $G$ by $H < G$, and that $H$ is a normal subgroup of $G$ 
    by $H \vartriangleleft G$. 
    
    If $H$ is a normal subgroup of $G$, and the order of $H$ is equal to the order of $G$, we write $H \trianglelefteq G$. 
\end{definition}

You should be very careful here. The equality $gH = Hg$ is a set equality. Not constants or numbers! It says that a 
right coset is equal to left a coset, it is not an equality elementwise.

\begin{example}
    Let $\mathbb{R}[x]$ denote the group of all polynomial with real coefficients under normal addition. 

    For any $f$ in $\mathbb{R}[x]$, let $f'$ denote the derivative of $f$. Then the mapping $f \to f'$ is a homomorphism from 
    $\mathbb{R}[x]$ to itself. The kernel of the derivative mapping is the set of all constant polynomials $f(x) = c$.
\end{example}

Now suppose we have a group $(G, cdot)$, and $H$ is a normal subgroup of $G$, just said $H \vartriangleleft G$. The set $G/H$ is defined by 
\[
G/H = \{ gH \> | \> g \in H \}
\]



\begin{theorem}[Orbit-Stabilizer theorem]
    For any group action $\phi : G \to \text{Permutation}(S)$, and for any $s \in S$,
    \begin{equation}
        |\text{Orb}(s)| \cdot |\text{Stab}(s)| = |G|.
    \end{equation}
\end{theorem}

\begin{theorem}
    The group of rotations of a cube is isomorphic to $S_4$.
\end{theorem}

\section{Group homomorphisms}

\begin{definition}
    A group homomorphism is a map $f: (G,{\color{red} \diamond}_G) \to (H, {\color{cyan} \bullet}_H)$ that respects binary operations:
    \begin{equation}
        f(a) \> {\color{cyan} \bullet}_H \> f(b) = f(a \> {\color{red} \diamond}_G \> b) \quad \forall a, b \in G
    \end{equation}
\end{definition}

\section{Tutorials}

\begin{mdframed}
    \vspace{-0.25cm}
    \hspace{-0.25cm}
    \begin{Exercise}
        Prove whether the following group $G$ together with operation $*$ is a group.
        \begin{enumerate}
            \item Let $*$ defined on $G = \mathbb{R}$ by letting $a * b = ab \quad \forall a, b \in \mathbb{R}$.
            \item Let $*$ defined on $G = 2\mathbb{Z}$ by letting $a * b = a + b \quad \forall a, b \in 2\mathbb{Z}$.
            \item Let $*$ defined on $G = \mathbb{R}^\times$ by letting $a * b = \sqrt{ab} \quad \forall a, b \in \mathbb{R}^\times$.
            \item Let $*$ defined on $G = \mathbb{Z}$ by letting $a * b = \max(a,b) \quad \forall a, b \in \mathbb{Z}$.
        \end{enumerate}
    \end{Exercise}

    \vspace{0.752cm}
    \begin{Exercise}
        Determine whether the given set of matrices under the specified operation, matrix addition or multiplication, is a group.
        \begin{enumerate}
            \item All $2 \times 2$ diagonal matrices under matrix addition.
            \item All $2 \times 2$ diagonal matrices under matrix multiplication.
            \item All $2 \times 2$ diagonal matrices with no zero diagonal entry under under matrix multiplication.
            \item All $2 \times 2$ diagonal matrices with all diagonal entries either $1$ or $-1$ under matrix multiplication.
            \item All $2 \times 2$ upper-triangular matrices under matrix multiplication.
            \item All $2 \times 2$ upper-triangular matrices under matrix addition.
            \item All $2 \times 2$ upper-triangular matrices with determinant $1$ under matrix multiplication.
            \item All $2 \times 2$ upper-triangular matrices with determinant either $1$ or $-1$ under matrix multiplication.
        \end{enumerate}
    \end{Exercise}

    \vspace{0.752cm}
    \begin{Exercise}
        Prove whether 
        \[
            G = \biggl\{ \begin{bmatrix}
            a & b \\ c & d
            \end{bmatrix} \bigg \vert \> ad - bc \neq 0,\> a,b,c,d \in \mathbb{Z} \biggr\}
        \]
        is a group under matrix multiplication.
    \end{Exercise}

    \vspace{0.752cm}
    \begin{Exercise}
        Prove whether 
        \[
            G = \biggl\{ \begin{bmatrix}
            a & b \\ 0 & d
            \end{bmatrix} \bigg \vert \> ad \neq 0,\> a,b,d \in \mathbb{Z} \biggr\}
        \]
        is a non-abelian group under matrix multiplication.
    \end{Exercise}

    \vspace{0.752cm}
    \begin{Exercise}
        Prove whether 
        \[
            G = \biggl\{ \begin{bmatrix}
            a & b \\ 0 & a^{-1}
            \end{bmatrix} \bigg \vert \> a \neq 0,\> a,b \in \mathbb{Z} \biggr\}
        \]
        is an abelian group under matrix multiplication.
    \end{Exercise}

    \vspace{0.752cm}
    \begin{Exercise}
        Let $(G, *)$ be a group and suppose that 
        \[
            a * b * c = e \quad \forall a,b,c \in G.
        \]

        Show that $b * c * a = e$.
    \end{Exercise}

    \vspace{0.752cm}
    \begin{Exercise}
        Show that if every element of the group $G$ is its own inverse, then $G$ is abelian.
    \end{Exercise}

    \vspace{0.752cm}
    \begin{Exercise}
        Show that every group with identity $e$ and $x \cdot x = x$ for all $x \in G$ is abelian.
    \end{Exercise}

    \vspace{0.752cm}
    \begin{Exercise}
        Show that if $G$ is a finite group with identity $e$ and with even number of elements, then there is an 
        $a \neq e$ in $G$ such that $a * a = e$.
    \end{Exercise}

    \vspace{0.752cm}
    \begin{Exercise}
        Suppose $G$ is a group such that 
        \[
            (ab)^2 = a^2\, b^2 \quad \forall a, b \in G.
        \]

        Show that $G$ is abelian.
    \end{Exercise}

    \vspace{0.752cm}
    \begin{Exercise}
        Find the order of the following cyclic groups.
        \begin{enumerate}
            \item The subgroup of $U(6)$ generated by $\displaystyle \cos \biggl( \frac{2\pi}{3}\biggr) + i \sin \biggl( \frac{2\pi}{3}\biggr)$.
            \item The subgroup of $U(5)$ generated by $\displaystyle \cos \biggl( \frac{4\pi}{3}\biggr) + i \sin \biggl( \frac{4\pi}{3}\biggr)$.
            \item The subgroup of $\mathbb{Z}/4\mathbb{Z} \times \mathbb{Z}/6\mathbb{Z}$ generated by $(1,5)$.
        \end{enumerate}
    \end{Exercise}

    \vspace{0.752cm}
    \begin{Exercise}
        Let $a$ and $b$ be elements of a group $G$. Show that if $ab$ has finite order $n$, then $ba$ also has order $n$.
    \end{Exercise}

    \vspace{0.752cm}
    \begin{Exercise}
        Show that a group with no proper nontrivial subgroup is cyclic. 
    \end{Exercise}

    \vspace{0.752cm}
    \begin{Exercise}
        Let $G$ be a nonabelian group with center $Z(G).$ Show that there exists an abelian subgroup $H$ of 
        $G$ such that $Z(G) \subset H$ but $Z(G) \neq H$. 
    \end{Exercise}

    \vspace{0.752cm}
    \begin{Exercise}
        Find all subgroups of the following groups and draw the subgroups diagram for the subgroups. 
        Hence, list all orders of the subgroups of the given groups.
        \begin{enumerate}
            \item $\mathbf{Z}_{36}$
            \item $\mathbf{Z}_{60}$
        \end{enumerate} 
    \end{Exercise}

    \vspace{0.752cm}
    \begin{Exercise}
        \begin{enumerate}
            \item Find all the proper nontrivial subgroups of $\mathbb{Z}_2 \times \mathbb{Z}_2 \times \mathbb{Z}_2$.
            \item Find all the subgroups of $\mathbb{Z}_2 \times \mathbb{Z}_4$ of order 4. 
        \end{enumerate}
    \end{Exercise}

    \vspace{0.752cm}
    \begin{Exercise}
        \begin{enumerate}
            \item Are the groups $\mathbb{Z}_2 \times \mathbb{Z}_{12}$ and $\mathbb{Z}_4 \times \mathbb{Z}_{6}$ isomorphic?
            \item Are the groups $\mathbb{Z}_8 \times \mathbb{Z}_{10} \times \mathbb{Z}_{24}$ and $\mathbb{Z}_4 \times \mathbb{Z}_{12} \times \mathbb{Z}_{40}$ isomorphic?
        \end{enumerate}
    \end{Exercise}

    \vspace{0.752cm}
    \begin{Exercise}
        Find the conjugacy classes of dihedral group $D_8$. 
    \end{Exercise}

    \vspace{0.752cm}
    \begin{Exercise}
        Show that a group that has only finite number of subgroups must be a finite group.
    \end{Exercise}

    \vspace{0.752cm}
    \begin{Exercise}
        Find all cosets of the subgroup $4\mathbb{Z}$ of $\mathbb{Z}$.
    \end{Exercise}

    \vspace{0.752cm}
    \begin{Exercise}
        Compute the quotient $\mathbb{Z}_{12}/ \langle 2 \rangle$.
    \end{Exercise}

    \vspace{0.752cm}
    \begin{Exercise}
        Show that if $H$ is a subgroup of index 2 in a finte group $G$, then every left coset of $H$ is also a 
        right coset of $H$.
    \end{Exercise}

    \vspace{0.752cm}
    \begin{Exercise}
        Let $\phi: G \to G$ be a mapping defined by 
                \[
                    \phi(x) = x^3 \quad \forall x \in G
                \]
                where $G = \mathbb{R} \setminus \{ 0 \}$ is a group defined under usual multiplication. Show that 
                $\phi$ is a homomorphism, and hence find $ker(\phi)$.
    \end{Exercise}

    \vspace{0.752cm}
    \begin{Exercise}
        Let $\phi: G \to G$ be a mapping defined by 
                \[
                    \phi(x) = 5^x \quad \forall x \in G
                \]
                where $G = \mathbb{R} \setminus \{ 0 \}$ is a group defined under usual multiplication. Show that 
                $\phi$ is a homomorphism, and hence find $ker(\phi)$.
    \end{Exercise}

    \vspace{0.752cm}
    \begin{Exercise}
        Let $G$ be a group and $g$ an element in $G$. Consider the mapping $\phi:G \to G$ defined as 
        $\phi(x) = gxg^{-1}$. Show that $\phi$ is an isomorphism.
    \end{Exercise}

    \vspace{0.752cm}
    \begin{Exercise}
        Find $ker(\phi)$ for map $\phi : \mathbb{Z}_{10} \to \mathbb{Z}_{20}$ such that 
        $\phi(1) = 8$.
    \end{Exercise}

    \vspace{0.752cm}
    \begin{Exercise}
        Find $ker(\phi)$ for map $\phi : \mathbb{Z} \times \mathbb{Z} \to \mathbb{Z} \times \mathbb{Z}$ such that 
        $\phi(1,0) = (2, -3)$ and $\phi(0,1) = (-1, 5)$.
    \end{Exercise}

    \vspace{0.752cm}
    \begin{Exercise}
        Let $\phi: G \to H$ be a group homorphism. Show that $\phi(G)$ is abelian if and only if
        \[
            xyx^{-1}y^{-1} \in ker(\phi) \quad \forall x, y \in G.
        \]
    \end{Exercise}

    \vspace{0.752cm}
    \begin{Exercise}
        Consider $A$ the set of affine maps of $\mathbb{R}$, that is 
        \[
            A = \{ f : x \mapsto ax + b, a \in \mathbb{R}^*, b \in \mathbb{R} \}
        \]
        \begin{enumerate}
            \item Show that $A$ is a group with respect to the composition of map.
            \item Let 
                \[
                    N = \{ g: x \mapsto x + b, b \in \mathbb{R} \}
                \]

                Show that $N \vartriangleleft A$.
            \item Show that the quotient group $A/N$ is isomorphic to $\mathbb{R}^*$.
        \end{enumerate}
    \end{Exercise}

    \vspace{0.752cm}
    \begin{Exercise}
        Let $G = S_4$ and let 
        \[
            H = \{ e, (1\> 2)(3 \> 4), (1\> 3)(2\> 4), (1\> 4)(2 \> 3)\}
        \]
        \begin{enumerate}
            \item Show that $H$ is a normal subgroup of $G$.
            \item Let $\overline{H} = \{ \sigma \in S_4 \> | \> \sigma(4) = 4 \}$. Define $\sigma : \overline{H} \to \text{Aut}(H)$
                by $\sigma(\tau) = \sigma \tau \sigma^{-1}$ for $\sigma \in \overline{H}$. Prove that 
                \[
                    \overline{H} \ltimes_\sigma H \cong S_4.
                \]
        \end{enumerate}
    \end{Exercise}

    \vspace{0.752cm}
    \begin{Exercise}
        Find (up to isomorphism) all abelian groups of order 45.
    \end{Exercise}

    \vspace{0.752cm}
    \begin{Exercise}
        Show that any group of order $p^2$ is abelian.
    \end{Exercise}

    \vspace{0.752cm}
    \begin{Exercise}
        Let $G$ be a group of order $pq$, where $p$ and $q$ are prime numbers. Show that every proper subgroup of $G$
        is cyclic.
    \end{Exercise}


    \vspace{0.752cm}
    \begin{Exercise}
        If $H, K \leq G$, show that $H \cap K \leq G$.
    \end{Exercise}

    \vspace{0.752cm}
    \begin{Exercise}
        If $N \vartriangleleft G$ and $H \leq G$, show that $NH \leq G$.
    \end{Exercise}

    \vspace{0.752cm}
    \begin{Exercise}
        If $N_1, N_2 \vartriangleleft G$, show that $N_1 \cap N_2 \vartriangleleft G$.
    \end{Exercise}


    \vspace{0.752cm}
    \begin{Exercise}
        If $N \vartriangleleft G$ and $H \leq G$, show that $H \cap N \vartriangleleft G$.
    \end{Exercise}
\end{mdframed}